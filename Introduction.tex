\chapter{Introduction}

The algebraic K-theory groups are fundamental invariants of rings. 
They encapsulate deep knowledge about the ring via its category of modules.
For rings of integers in number fields, the K-theory groups contain information about the class group and group of units of the ring, the Brauer group of the field and values of its Dedekind zeta function.
Algebraic K-theory can also be applied to schemes. In fact, it was first defined in this context by Grothendieck during the work on his Riemann-Roch theorem, where he discovered a remarkable relationship to algebraic cycles.
Later, Bloch pushed this relationship further to his newly defined higher Chow groups.
There are also spectacular applications towards geometric topology.
Starting with the early work of Wall and the s-cobordism theorem by Barden-Mazur-Stalling it culminated in the stable parametrized h-cobordism theorem by Waldhausen, Jahren and Rognes. It says that the stable h-cobordism space of say a smooth manifold $M$ can be obtained from the K-theory space of $\S[\Omega M]$, the spherical group ring of the loop space of $M$.
This is remarkable because this allows us - as does the classical h-cobordism theorem - to obtain geometric results for $M$ purely using homotopy theory.
One of its implications is that we can calculate many homotopy groups of the diffeomorphism group of $M$ via the K-theory groups of $\S[\Omega M]$.
For example, the homotopy groups of the diffeomorphism group of a disc $\pi_* (\operatorname{Diff}(D^n))$ can (in a range depending on $n$) be obtained by computing the K-theory groups of the sphere spectrum. This boils down to the understanding of $K_*(\Z)$ about which we know a lot using motivic homotopy theory and several major results in number theory that allow us to compute certain étale cohomology groups.
\\
These powerful results and theorems now present us with the challenge to actually compute K-theory groups. 
This problem is in general very hard! 
There are several successful approaches towards these computations, which nicely complement each other. Among them are motivic methods, controlled algebra and trace methods. 
Let us only describe the last one, as it is the one of relevance for this thesis.
The idea of trace methods is to consider other - ideally more computable - spectrum valued invariants of rings and compare K-theory to them via so-called trace maps. 
The hope is that while these invariants are more computable, the difference to K-theory is not too big. In other words, the trace map is close to an isomorphism. 
The most successful of these comparisons is with the trace map $K\xrightarrow{\tr} \TC$ to the so-called \textit{topological cyclic homology} introduced by Bökstedt-Hsiang-Madsen.
By work of Dundas-Goodwillie-McCarthy and Hesselholt-Madsen, we now know that the trace map is a remarkably close approximation that has been used to calculate K-theory in many cases.
Due to the recent work of Nikolaus-Scholze, topological cyclic homology itself can essentially be computed via two spectral sequences out of yet another invariant: Topological Hochschild homology $\THH$, which is the main player of this thesis. \\
Let us now describe the problem we want to solve.
We will specifically study (topological) Hochschild homology of complete discrete valuation rings of mixed characteristic with perfect residue field of characteristic $p$. 
To be concrete, all of these rings are certain extensions of the $p$-adic integers, like $\Z_p[\sqrt[n]{p}]$ or $\Z_p[\zeta_{p^n}]$. They arise for example as completions of rings of integers in number fields and are thus of fundamental importance in algebraic number theory.
The topological Hochschild homology of them has already been computed in the work of Lindenstrauss-Madsen. 
Recently Krause and Nikolaus gave a more conceptual proof of the same result, which we present in Chapter \ref{THHofCDVR} of this thesis. 
Unfortunately, both approaches do not give a \textit{functorial} computation. They only identify $\THH$ of these rings but do not answer the question what it does to morphisms. 
This is unsatisfactory for at least two reasons: We might want to understand the action of the Galois group on K-theory. Because the trace map $K\xrightarrow{\tr}\TC$ is a natural transformation and $\TC$ is functorially obtained from $\THH$, the first step to understand the functoriality of K-theory is to understand it for $\THH$.
Secondly, functorial descriptions are often needed for further computations. For example, in our computation, we crucially need that we not only know the topological Hochschild homology groups of $\Z_p[z]$ but \textit{also} how endomorphisms of $\Z_p[z]$ act on the topological Hochschild homology groups of it. \\
We do not succeed to describe the functoriality of $\THH$ for this class of rings in complete generality. Instead, we obtain three partial results, which we explain in the three sections of Chapter \ref{FunctorialityofTHHforCDVRS}\\
In Section \ref{HHFunctoriality}, we give a fully functorial description of ($p$-completed) Hochschild homology for CDVRs. We achieve this using the Hochschild-Kostant-Rosenberg filtration and by carefully keeping track of the functoriality in the setting of a relative computation. Since the homotopy groups of $\THH(R)_p^{\wedge}$ and $\HH(R)_p^{\wedge}$ are naturally isomorphic in degrees less than $2p-1$, we thus also give a functorial description of the low homotopy groups of $\THH$.\\
In the second section, we describe an approach to understanding the functoriality using the Tor spectral sequence. 
For this, we use that by a general classification result all mixed characteristic discrete valuation rings are obtained from $\Z_p$ by adjoining a root of an Eisenstein polynomial, i.e. $R=\Z_p[z]/E(z)$ for $E(z)\in \Z_p[z]$ Eisenstein\footnote{Actually, we need to take the Witt vectors of the (perfect) residue field of $R$, but for the sake of exposition we stick to the easiest case that the residue field is $\F_p$ and the Witt vectors are $W(\F_p)=\Z_p$}. 
This allows us to write $R$ as $R=\Z_p[z]\otimes_{\Z_p[z]}\Z_p$, which gives us $\THH(R)=\THH(\Z_p[z])\otimes_{\THH(\Z_p[z])}\THH(\Z_p)$
%where $\Z_p[z]$ acts on the left factor by the map $z\mapsto E(z)$ and on the right factor by $z\mapsto 0$. Since $\THH$ preserves tensor products of commutative rings, we get $\THH(R)=\THH(\Z_p[z])\otimes_{\THH(\Z_p[z])}\THH(\Z_p)$. We further have an equivalence $\THH(\Z_p[z])\simeq\THH(\Z_p)\otimes_{\Z_p}\HH(\Z_p[z]/\Z_p)$. This is not a natural equivalence on the level of spectra, but it is natural after taking homotopy groups $\THH_*(\Z_p[z])\simeq\THH_*(\Z_p)\otimes_{\Z_p}\Omega^*_{\Z_p[z]/\Z_p}$. Since we are actually interested in $\THH(R)$, this motivates us to contemplate the Tor spectral sequence for the tensor product $\THH(R)=\THH(\Z_p[z])\otimes_{\THH(\Z_p[z])}\THH(\Z_p)$. We know the $E^2$-page, which is given by certain graded Tor-groups, and we might hope to understand the differentials and extension problems.
We can then use the Tor spectral sequence associated to this tensor product to try to understand $\THH_*(R)$.  \\
Unfortunately, this approach has two problems. First of all, we are only able to understand the effect of maps between CDVRs, which we can 'lift' to the level of presentations $\Z_p[r]\xrightarrow{r\mapsto E(z)} \Z_p[z]\rightarrow R$. Secondly, while we can compute the $E^2$-page and understand several things about the differentials and extension problems, there are still many indeterminacies left which make it hard to give a concrete result. But we can for example say something in the tamely ramified setting .\\
In the last section, we give another partial result that works for all complete discrete valuation rings with perfect residue field, but only for very special maps. Specifically, we can only deal with 'monomial' maps, i.e. those which send one uniformizer to a power of the other uniformizer. 
%\todo{Describe the three sections}


% Let us now give a short outline of our approach to understand the functoriality of $\THH_*(R)$ (more precisely we deal with the $p$-completions of these groups).
% Firstly we use, that by a general classification result all mixed characteristic discrete valuation rings are obtained from $\Z_p$ by adjoining a root of an Eisenstein polynomial, i.e. $R=\Z_p[z]/E(z)$ for $E(z)\in \Z_p[z]$ Eisenstein\footnote{Actually we need to take the Witt vectors of the residue field of $R$, but for sake of exposition we stick to the easiest case that the residue field is $\F_p$ and the Witt vectors are $W(\F_p)=\Z_p$}.
% This allows us to write $R$ as follows $R=\Z_p[z]\otimes_{\Z_p[z]}\Z_p$, where $\Z_p[z]$ acts on the left factor by the map $z\mapsto E(z)$ and on the right factor by $z\mapsto 0$. 
% Now we can observe, that $\THH$ preserves tensor products of commutative rings, i.e. we get $\THH(R)=\THH(\Z_p[z])\otimes_{\THH(\Z_p[z])}\THH(\Z_p)$. This is progress because we know what $\THH(\Z_p)$ is. Furthermore, by a Hochschild-Kostant-Rosenberg type result we also obtain $\THH(\Z_p[z])\simeq\THH(\Z_p)\otimes_{\Z_p}\HH(\Z[z]/\Z)$. This is not a natural equivalence on the level of spectra, but it is natural after taking homotopy groups $\THH_*(\Z_p[z])\simeq\THH_*(\Z_p)\otimes_{\Z}\Omega^*_{\Z[z]/\Z}$.
% We exactly know how Kähler differentials are functorial, thus we completely understand the functoriality on the level of homotopy groups. 
% Now we want to go from these homotopy groups to $\THH_*(R)$ and for this we can employ the Tor-spectral sequence to the tensor product formula, we found above: $\THH(R)=\THH(\Z_p[z])\otimes_{\THH(\Z_p[z])}\THH(\Z_p)$. This spectral sequences has $E^2$-page given by graded Tor-groups and converges to $\THH_*(R)$. 
% A map $R\to S$ induces a map between the respective spectral sequences. Therefore we need to understand the differentials and extension behaviour. 
% This is in fact possible, because we know by the result chapter \ref{THHofCDVR}, what $\THH(R)$ is and thus the target of the spectral sequence.


% Besides these results, we also have a chapter on preliminaries, which gives references to the notions from higher category theory and higher algebras that we need. 
% In the appendix, we recall the definition of algebraic K-theory. 



\section*{Acknowledgements}
First and foremost, I would like to thank Thomas Nikolaus for taking me on as an external student and giving me a taste of what cutting edge research in mathematics consists of as well as being an incredible role model regarding dedication and passion for mathematics. \\
Secondly, I would like to thank Gabriela Weitze-Schmithüsen for being the local supervisor at my home university. I am also thankful to her for introducing me to category theory in a very early stage of my studies, which greatly influenced my structural thinking about mathematics and certainly paved a part of the way towards the topic of this thesis. \\
I would also like to thank Achim Krause, Felix Janssen and Jonas McCandless for teaching me large parts of my knowledge of homotopy theory. Without you, this thesis could never exist in the present form.
\\
\indent
Last but certainly not least, I would like to express my deep gratitude for the unwavering support of my parents and girlfriend during the process of writing this thesis.