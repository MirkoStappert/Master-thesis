\appendix
\chapter{K-Theory}
In this appendix we give a brief review of the definition of algebraic K-theory for (connective) ring spectra.
Let $R$ be a connective ring spectrum.
An $R$-module spectrum $M$ is called finitely generated projective, if there is another $R$-module $N$ and  $n\in\N$ such that $M\oplus N=R^n$.
Denote by $\Proj_R\subset \Mod_R$ the (full) subcategory of finitely generated projective $R$-module spectra. This is becomes a symmetric monoidal $\infty$-category under direct sum.
\begin{itemize}
    \item Definition of category of projective modules (or rather groupoid core) 
    \item perhaps also $\Perf_R$= smallest stable subcategory of $\Mod_R$ containing $R$ and which is closed under retracts. (This has structure of a Waldhausen $\infty$-category and can define K-theory for this \cite{barwickK-theory})
    \item This provides an spaces with $\Einf$-structure because $\Proj_R$ is symmetric monoidal
    \item adjunction between grouplike $\Einf$-spaces and $\Einf$-spaces 
    \item Equivalence of categories between connective spectra and grouplike $\Einf$-spaces
    \item Define $K(R)$ as connective spectrum corresponding to group completion of groupoid core of $\Proj_R$.
    \item Perhaps mention that it is a ring spectrum (if $R$ is) cite Barwick/BGT/Nikolaus \cite[Section~8]{GGNuniversalityloop} \cite{BGTuniquenessmulti}
\end{itemize}