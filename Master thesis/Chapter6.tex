\chapter{Functoriality for THH of CDVRs}
In this chapter, we will investigate functorial computations. In the first section we will warm-up with Hochschild homology. In the second section we will employ the Tor-spectral sequence and use its functoriality to give a description of the functoriality for $\THH$. We can relate a certain part of this spectral sequence with a similar spectral sequence for $\HH$. \\ In the third and last section we will discuss a 'partial' functoriality of the descent spectral sequence from section 3.4, which allows us to compute $\THH$ for certain maps. We will then compare these computations to the ones of section two.\\
See \cite[~Proposition 7.2.1.19]{lurie2017higher} for a statement of the Tor spectral sequence for module spectra over a ring spectrum 
or \cite[\href{https://stacks.math.columbia.edu/tag/061Y}{Tag 061Y}]{stacks-project} for the more classical result in the case of chain complexes.
\\
Let $R$ be a mixed characteristic complete discrete valuation ring with perfect residue field. The following are the main steps in our computation in section two.
\begin{enumerate}
    \item Write $R$ as a pushout $R\cong\Z_p[z]\otimes_{\Z_p[z]}\Z_p$ via action through the Eisenstein polynomial (see footnote 5)
    \item $\THH$ preserves this pushout because the involved rings are commutative, and for $\Einf$-rings, $\THH$ is given by the colimit over $S^1$ in $\CAlg$ which means that it commutes with arbitrary colimits.
    \item Use the fact that $\THH(\Z_p[z])\simeq \THH(\Z_p)\otimes_\Z \HH(\Z[z]/\Z)$ but not naturally. But have natural isomorphism on homotopy groups
    $\THH*(\Z_p[z])\simeq \THH_*(\Z_p)\otimes_\Z \Omega_{\Z[z]/\Z)}$.
    \item Now employ Tor-spectral sequence:\\
     $E^2_{i,j}=\Tor_{\pi_*(R)}^i\left( \pi_*(M), \pi_*(M)\right)_{(j)}\Rightarrow \pi_{i+j}\left( M\otimes_R N \right)$
    , i.e. in our case: 
    \\
    $\Tor_{\THH_*(\Z_p[z])}^i\left( \THH_*(\Z_p[z]),\THH_*(\Z_p)\right)_{(j)}\Rightarrow \THH_{i+j}(R)$ (or rather the $p$-completed version)
    \item Analyze the differentials and extension problems in the spectral sequence.
\end{enumerate}



%%%%%%%%%%%%%%%%%%%%%%%%%%%%%%%%%%%%%%%%%%
\section{Functoriality of $\HH$ for CDVRs}
%%%%%%%%%%%%%%%%%%%%%%%%%%%%%%%%%%%%%%%%%%
\label{HHfunctoriality}
In this section we will describe the functoriality of Hochschild homology. To do this, we will need to make use of nonabelian derived functors. They are a generalization of ordinary derived functors, which have the advantage that they also work in non-additive settings, i.e. for non-additive categories or non-additive functors. This originally goes back to \cite{DoldPuppeHomologie}, for a modern formulation see \cite[Section~5.5.8]{HigherToposTheory}.
The main idea is to use simplicial resolutions, which make sense in every category, instead of resolutions by chain complexes, which are only sensible in an additive setting. 
We will only need the concept to derived the non-additive exterior product functor $\Lambda^n_R:\Mod_R\to \Mod_R$, for $R$ an ordinary ring and $\Mod_R$ the 1-category of $R$-modules. The nonabelian derived functor is then a functor $L\Lambda^n_R:D(R)_{\geq 0}\to D(R)_{\geq 0}$. 
The general recipe to compute it goes as follows\footnote{In general we would need to resolve by finitely generated, projective modules, but since $\Lambda_R^n$ preserves filtered colimits this is not necessary in out case.}: Take a complex (e.g. an ordinary module concentrated in degree 0) find a resolution by a chain complex of projective $R$-modules, use the Dold-Kan correspondence (see e.g. \cite[Section~1.2.3]{lurie2017higher}) to turn this complex into an simplicial $R$-module. 
Then apply the functor $\Lambda^n_R$ levelwise to this simplicial object and finally convert it back into a chain complex using the inverse of the Dold-Kan correspondence. \\
The difference to ordinary derived functors only consists in the fact that we apply the functor that we want to derive \textit{after} turning the projective resolution into an simplicial object. 
As in the ordinary case the value does not depend on the choice of projective resolution.
\begin{enumerate}
    \item Write down pushout square, apply HH to it
    \item Need to compute derived TP, 
    \item Resolve $\Z_p$ over exterior algebra $\Lambda_{\Z_p}(e)$ gives divided power algebra
    \item TP then gives same DP algebra but with different differential
    \item Compute Homology of this, concentrated in odd degrees and given by the same group
    %\item But different functoriality, power.
\end{enumerate}

\begin{lem}
    For all rings in consideration the cotangent complex relative to $\Z_p$ is concentrated in degree 0 and given by the Kähler differentials
    \begin{equation*}
        \bL_{R/\Z_p}=\Omega^1_{R/\Z_p}[0].
    \end{equation*}
\end{lem}
\begin{proof}
    Using Proposition \ref{description mixed char CDVR}, we get a map $\Z_p[z]\to R\simeq \Z_p[z]/E$. 
    The maps $\Z_p\to \Z_p[z]\to R$ then induce the fiber sequence
    \begin{equation*}
        \bL_{\Z_p[z]/\Z_p}\otimes_{\Z_p[z]}R\to \bL_{R/\Z_p}\to \bL_{R/\Z_p[z]}.
    \end{equation*}
    By general facts about the cotangent complex, we know that $\bL_{\Z_p[z]/\Z_p}=\Omega^1_{\Z_p[z]/\Z_p}[0]$ (this is a free $\Z_p[z]$-modules, so we do not need to derive the tensor product) and $\bL_{R/\Z_p[z]}=(E)/(E))^2[1]$.
    Since they are concentrated in degree 0 and 1 respectively, $\bL_{R/\Z_p}$ can also only have homology in degrees 0 and 1. To prove the statement we thus have to check, that $H_1(\bL_{R/\Z_p})=0$. Let us look at the long exact sequence associated to the fiber sequence.
    \begin{equation*}
        0=H_1(\bL_{\Z_p[z]/\Z_p})\to H_1(\bL_{R/\Z_p})\to (E)/(E)^2 \xrightarrow{\delta} \Omega^1_{\Z_p[z]/\Z_p}\otimes_{\Z_p[z]}R.
    \end{equation*} 
    The map $\delta$ sends the class of $E$ to $E'dz\otimes 1$ and is thus injective, which implies that $H_1(\bL_{R/\Z_p})=0$.
\end{proof}
Section 2.1 of the thesis \cite{OnderiveddeRhamcohomoology} contains this example and many more as well as a recollection of the necessary tools to compute the cotangent complex.

\begin{prop}
    For all rings $R$ in consideration we have the natural isomosphism 
    \begin{equation*}
        \HH_{2n-1}(R;\Z_p)=\HH_{2n-1}(R/\Z_p)=H_{n-1}(L\Lambda^{n}_R \Omega^1_{R/\Z_p}).
    \end{equation*} 
\end{prop}
\begin{proof}
    We will use the HKR filtration on $\HH(R/\Z_p)$, in particular the spectral sequence that this filtration induces. The associated graded of the HKR filtration is given by $\gr^i=L\Lambda^i_R\bL_{R/\Z_p}[i]$. By the previous Lemma, we have $\bL_{R/\Z_p}=\Omega^1_{R/\Z_p}$.
    The associated spectral sequence thus takes the form
    \begin{equation*}
        E^1_{i,j}=\pi_i(L\Lambda^j_R(\Omega^1_{R/\Z_p}))\Rightarrow \HH_{i+j}(R/\Z_p)
    \end{equation*}
    Because we have a two term resolution of $\Omega^1_{R/\Z_p}$, the simplicial $R$-module $L\Lambda^j_R(\Omega^1_{R/\Z_p})$ has only degenerate simplices in dimension $j$ and higher. Therefore $H_i(L\Lambda^j_R(\Omega^1_{R/\Z_p}))=0$ for $j\geq i$\todo{or is it i-1?}
    But also all the lower homology groups necessarily vanish, because $\Lambda_R^i(R^j)=0$ for $j<i$. \\
    Hence we can see, that the $E^1$-page of the spectral sequence is (besides an $R$ in degree $(0,0)$) concentrated on the line $(n+1,n)$, where it is given by $H_{n-1}(L\Lambda^{n}_R \Omega^1_{R/\Z_p})$.
    All differentials are necessarily 0 and the spectral sequence degenerates on the $E^1$-page with no room for extension problems.
\end{proof}
Using this and computing the nonabelian derived functor of the Kähler differentials.


\begin{lem}
    For the rings in consideration we have
    \begin{equation*}
        \HH_{2n-1}(R;\Z_p)=\Omega^1_{R/\Z_p}\simeq R/E'(\pi).
    \end{equation*}
\end{lem}
\begin{proof}
    By the previous lemma, we want to compute $H_{n-1}(L\Lambda^{n}_R \Omega^1_{R/\Z_p})$. We will use the recipe, explained at the beginning of the section. Thus we need a projective resolution of the $R$-module $\Omega^1_{R/\Z_p}$, which is conveniently provided by 
    \begin{equation*}
       R\xrightarrow{\cdot E'(\pi)} R\to \Omega^1_{R/\Z_p}. 
    \end{equation*}
    Using Dold-Kan, we get a simplicial $R$-module with $R^{n+1}$ in degree $n$. Applying $\Lambda_R^n$ levelwise and converting it back into gives a chain complex again. We want to compute $H_{n-1}$ of it, so we only need the degrees $n-2,n-1,n$, which look as follows 
    \begin{equation*}
        \Lambda_R^nR^{n-1}=0\la \Lambda_R^n R^n=R \xleftarrow{d} \Lambda_R^n R^{n+1}=R^{n+1}. 
    \end{equation*} 
    Chasing through the Dold-Kan construction we see that the differential is given by multiplication with $E'(\pi)$. Thus $H_{n-1}\left( L\Lambda_R^n(\Omega^1_{R/\Z_p}) \right)=R/E'(\pi)$, which by the last lemma is exactly $\HH_{2n-1}(R;\Z_p)$.
\end{proof}
Furthermore, we obtain the following Proposition, which describes the functoriality of Hochschild homology.
\begin{prop}
    Let $R\xrightarrow{\varphi} S$ be a map of CDVRs. We get an induced map $\mathrm{d}\varphi:\Omega^1_{R/\Z_p}\to \Omega^1_{S/\Z_p}$. Choose uniformizers with respective Eisenstein polynomials $E,\tilde{E}$. These Eisenstein polynomials give a resolution of the Kähler differentials and by the fundamental theorem of homological algebra we can lift $\mathrm{d}\varphi$ to maps $f$ and $g$ on the terms of the resolutions. 
    
    
    % https://q.uiver.app/?q=WzAsNixbMSwwLCJSIl0sWzIsMCwiXFxPbWVnYV4xX3tSL1xcbWF0aGJie1p9X3B9Il0sWzAsMCwiUiJdLFswLDEsIlMiXSxbMSwxLCJTIl0sWzIsMSwiXFxPbWVnYV4xX3tTL1xcbWF0aGJie1p9X3B9Il0sWzAsMV0sWzIsMCwiXFxjZG90IEUoXFxwaSkiXSxbMyw0LCJcXGNkb3QgXFx0aWxkZXtGfShcXHRpbGRle1xccGl9KSJdLFs0LDVdLFsxLDUsIlxcbWF0aHJte2R9XFx2YXJwaGkiXSxbMCw0LCJmIiwwLHsic3R5bGUiOnsiYm9keSI6eyJuYW1lIjoiZGFzaGVkIn19fV0sWzIsMywiZyIsMCx7InN0eWxlIjp7ImJvZHkiOnsibmFtZSI6ImRhc2hlZCJ9fX1dXQ==
\[\begin{tikzcd}
	R & R & {\Omega^1_{R/\mathbb{Z}_p}} \\
	S & S & {\Omega^1_{S/\mathbb{Z}_p}}
	\arrow[from=1-2, to=1-3]
	\arrow["{\cdot E(\pi)}", from=1-1, to=1-2]
	\arrow["{\cdot \tilde{E}(\tilde{\pi})}", from=2-1, to=2-2]
	\arrow[from=2-2, to=2-3]
	\arrow["{\mathrm{d}\varphi}", from=1-3, to=2-3]
	\arrow["f", dashed, from=1-2, to=2-2]
	\arrow["g", dashed, from=1-1, to=2-1]
\end{tikzcd}\]
    The map on Hochschild homology groups $\HH_{2n-1}(R;\Z_p)\to         \HH_{2n-1}(S;\Z_p)$ is now induced by the map $fg^{n-1}:R\to S,\ a\mapsto f(a)g(a)^{n-1}$.
\end{prop}
\begin{proof}
    This is simply a matter of understanding the functoriality of the statement in the last lemma. We get the diagram
    % https://q.uiver.app/?q=WzAsNixbMSwwLCJcXExhbWJkYV9SXm5SXm4iXSxbMiwwLCIwIl0sWzEsMSwiXFxMYW1iZGFfU15uU15uIl0sWzIsMSwiMCJdLFswLDEsIlxcTGFtYmRhX1NeblNee24rMX0iXSxbMCwwLCJcXExhbWJkYV9SXm5SXntuKzF9Il0sWzAsMV0sWzIsM10sWzQsMiwiZCJdLFs1LDAsImQiXSxbNSw0XSxbMCwyXV0=
    \[\begin{tikzcd}
	{\Lambda_R^nR^{n+1}} & {\Lambda_R^nR^n} & 0 \\
	{\Lambda_S^nS^{n+1}} & {\Lambda_S^nS^n} & 0
	\arrow[from=1-2, to=1-3]
	\arrow[from=2-2, to=2-3]
	\arrow["d", from=2-1, to=2-2]
	\arrow["d", from=1-1, to=1-2]
	\arrow[from=1-1, to=2-1]
	\arrow[from=1-2, to=2-2]
    \end{tikzcd}\]
    The relevant map $\Lambda_R^nR^n\to \Lambda_S^nS^n$ is induced by the map $R^n\to S^n$ given by the diagonal matrix $diag(g,\dots, g, f)$. This matrix has determinant $fg^{n-1}$, which is thus the induced map on homology.
\end{proof}
%%%%%%%%%%%%%%%%%%%%%%%%%%%%%%%%%%%%%%%%%%%%%%%%%%%%%%%%%%%%%%%
\section{Functoriality of $\THH$ via the Tor spectral sequence}
%%%%%%%%%%%%%%%%%%%%%%%%%%%%%%%%%%%%%%%%%%%%%%%%%%%%%%%%%%%%%%%
We will crucially use Lemma \ref{description mixed char CDVR} to write all our rings as quotients of $\Z_p[z]$. This allows us to deduce our computations from the computation for $\THH(\Z_p[z])$ and to understand the functoriality of $\THH(\Z_p[z])$ under self maps of $\Z_p[z]$. We do not understand the functoriality of $\THH(\Z_p[z])$ on the level of spectra, but for homotopy groups we have a relative Hochschild-Kostant-Rosenberg style result.
\begin{lem}
    Let $R$ be an ordinary ring. Then we have the following expression for the topological Hochschild homology of the polynomial ring over $R$.
    \begin{equation*}
        \THH(R[z])\simeq \THH(R)\otimes_\Z \HH(\Z[z]).
    \end{equation*}
    This is not a natural equivalence. But the resulting isomorphism on homotopy groups $\THH_*(R[z])\simeq \THH_*(R)\otimes_\Z \Omega_{\Z[z]/\Z}$ is a natural isomorphism.
\end{lem}
\begin{proof} Using the basic properties of $\THH$ and the equivalence $R[z]\simeq R\otimes_\S \S[z]$ we obtain
    \begin{align*}
        \THH(R[z])\simeq \THH(R\otimes_\S \S[z])&\simeq\THH(R)\otimes_\S \THH(\S[z]) \\
        &\simeq \THH(R)\otimes_\Z (\Z\otimes \THH(\S[z])) \\
        &\simeq\THH(R)\otimes_\Z (\THH(\Z\otimes \S[z]/\Z))\\
        &\simeq 
         \THH(R)\otimes_\Z \HH(\Z[x])
    \end{align*}\qedhere
\end{proof}
In particular we have a natural isomorphism $\THH(\Z_p[z])\simeq \THH(\Z_p)\otimes_\Z \HH(\Z[z])$. Taking the $p$-completion yields 
\begin{equation*}
    \THH(\Z_p[z];\Z_p)\simeq \THH(\Z_p;\Z_p)\otimes_{\Z_p} \HH(\Z_p[z]/\Z_p).
\end{equation*} 
We do not need to $p$-complete on the outside of the tensor product, because ...\todo{Write reason}.\\
Let us now apply this result. For this and for the rest of the section let $R$ be a mixed-characteristic complete discrete discrete valuation ring with perfect residue field of characteristic $p$. By Lemma \ref{description mixed char CDVR} we can thus find an Eisenstein polynomial $E(z)$ such that
\begin{equation*}
    R=\Z_p[z]/E(z)=\Z_p[z]\otimes_{\Z_p[z]}\Z_p,
\end{equation*}
with actions prescribed by the maps $\Z_p[z]\ra \Z_p[z], z\mapsto E(z)$ and $\Z_p[z]\to \Z_p, z\mapsto 0$. \\
Using the monoidality of $\THH$ (Proposition 2.6) we therefore get 
\begin{equation*}
    \THH(R)=\THH(\Z_p[z])\otimes_{\THH(\Z_p[z])}\THH(\Z_p).
\end{equation*}
To $p$-complete the left hand side, we again only have to $p$-complete all the involved terms and not on the outside because ...\todo{Write reason}. We then know all the homotopy groups of the terms involved on the right side, because we know $\THH(\Z_p[z];\Z_p) $ and $\THH(\Z_p;\Z_p)$. But we of course want to know the homotopy groups of $\THH(R;\Z_p)$ and for this we can use the Tor-spectral sequence. 
It takes the form
\begin{equation*}
    E^2_{i,j}=\Tor_{\THH_*(\Z_p[z];\Z_p)}^i\left( \THH_*(\Z_p[z];\Z_p),\THH_*(\Z_p;\Z_p)\right)_{(j)}\Rightarrow \THH_{i+j}(R;\Z_p)
\end{equation*}
\textbf{The $E^2$-page only contains terms we functorially know}.

%This will allow us to compute the effect of $\THH$ on maps $R\to S$.
Since the action of $\THH_*(\Z_p[z];\Z_p)$ on $\THH_*(\Z_p;\Z_p)$ factors through the action of $\HH(\Z_p[z];\Z_p)=\Omega^*_{\Z_p[z]/\Z_p}$ on $\Z_p$, it suffices to resolve $\Z_p$ as a $\Omega^*_{\Z_p[z]/\Z_p}$-module. A resolution can be provided by a free divided power algebra on an exterior algebra over $\Omega^*_{\Z_p[z]/\Z_p}$, i.e. we have a quasi-isomorphism of graded $\Omega^*_{\Z_p[z]/\Z_p}$-modules
\begin{equation*}
    \Z_p \lai \Gamma_{\Lambda_{\Omega^*_{\Z_p[z]/\Z_p}}(b)}\{a\}, \ |a|=(1,0), \ |b|=(1,1), \ \partial(a)=dz, \ \partial(b)=z.
\end{equation*}
Explicitely this looks as follows:
\begin{align*}
    \Z_p\la \Omega^*_{\Z_p[z]/\Z_p}\la \Omega^*_{\Z_p[z]/\Z_p}\{a,b\}\la \Omega^*_{\Z_p[z]/\Z_p}\{a^{[2]}, ba\}\la \dots 
\end{align*}
with the maps (from left to right) given by  $[0\mapsfrom z, dz]$, $[z\mapsfrom b, dz\mapsfrom a]$ and $[az-bdz \mapsfrom ba, adz \mapsfrom a^{[2]}]$.
Let us abbreviate $\Gamma\coloneqq \Gamma_{\Lambda_{\Omega^*_{\Z_p[z]/\Z_p}}(b)}\{a\}$. The $E^2$-page is then the homology of
\begin{equation*}
    \THH(\Z_p;\Z_p)\otimes 
        \left(   
        \Omega^*_{\Z_p[z]/\Z_p} \otimes_{\Omega^*_{\Z_p[z]/\Z_p}} \Gamma   
        \right).
\end{equation*} 
The tensor product $\Omega^*_{\Z_p[z]/\Z_p} \otimes_{\Omega^*_{\Z_p[z]/\Z_p}}-$ has only the effect, that it changes the differential to $\partial(a)=E'(z)dz, \ \partial(b)=E(z)$. Hence to fully compute the $E^2$-page, we must identify the homology of $\Gamma$ with the modified differential and then stack infinitely many copies of the homology and the derived mod $p^n$ reduction on top of each other.
%%%%%%%%%%%%%%%%%%%%%%%%%%%%%%%%%%%%%%%%%%%%%%%%%%%%%%%%%%%%%%%%%%%
\section{(Partial) Functoriality of $\THH$ via the descent spectral sequence}
%%%%%%%%%%%%%%%%%%%%%%%%%%%%%%%%%%%%%%%%%%%%%%%%%%%%%%%%%%%%%%%%%%%
Have functoriality in monomial maps, i.e. $\S[z]\to \S[z],\  z\mapsto z^n$, induced by the maps $\N\xrightarrow{\cdot n} \N$, that provide squares. This applies for example in the cases of the inclusions $\Z_p\hookrightarrow \Z_p[\sqrt[n]{p}]$ or $\Z_p[\sqrt[k]{p}]\hookrightarrow \Z_p[\sqrt[kn]{p}]$. \\
Using the base change formula 
\begin{equation*}
    \THH(R/\S[z])=\THH(R)\otimes_{\THH(\S{z]})}\S[z]
\end{equation*}
We obtain the following map of pushout squares and thus get the dashed induced map.
% https://q.uiver.app/?q=WzAsOCxbMCwzLCJcXFRISChcXFNbel0pIl0sWzIsMywiXFxUSEgoUi9cXFNbel0pIl0sWzAsMSwiXFxUSEgoUikiXSxbMiwxLCJcXFNbel0iXSxbMSwwLCJcXFRISChTKSJdLFszLDAsIlxcU1t3XSJdLFszLDIsIlxcVEhIKFMvXFxTW3ddIl0sWzEsMiwiXFxUSEgoXFxTW3ddKSJdLFswLDFdLFsyLDBdLFsyLDNdLFszLDFdLFsyLDRdLFszLDVdLFsxLDYsIiIsMCx7InN0eWxlIjp7ImJvZHkiOnsibmFtZSI6ImRhc2hlZCJ9fX1dLFs1LDZdLFswLDddLFs3LDZdLFs0LDddLFs0LDVdXQ==
\[\begin{tikzcd}
	& {\THH(S)} && {\S[w]} \\
	{\THH(R)} && {\S[z]} \\
	& {\THH(\S[w])} && {\THH(S/\S[w])} \\
	{\THH(\S[z])} && {\THH(R/\S[z])}
	\arrow[from=4-1, to=4-3]
	\arrow[from=2-1, to=4-1]
	\arrow[from=2-1, to=2-3]
	\arrow[from=2-3, to=4-3]
	\arrow[from=2-1, to=1-2]
	\arrow[from=2-3, to=1-4]
	\arrow[dashed, from=4-3, to=3-4]
	\arrow[from=1-4, to=3-4]
	\arrow[from=4-1, to=3-2]
	\arrow[from=3-2, to=3-4]
	\arrow[from=1-2, to=3-2]
	\arrow[from=1-2, to=1-4]
\end{tikzcd}\]