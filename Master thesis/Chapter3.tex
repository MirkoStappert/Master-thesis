\chapter{Topological Hochschild homology and cyclotomic spectra}
Structures deduced from cyclotomic structure on $\THH(A)$.
\\
Define $\THH$ and the cyclotomic structure on it. 


\begin{defn}\cite[Chapter~2.1]{NS}
a) For a fixed prime p, a \textbf{$p$-cyclotomic spectrum} is a spectrum with $\pg$-action together with a $\pg$-equivariant map $\varphi_p: X \ra X^{tC_p}$ \newline
b) A \textbf{cyclotomic spectrum} is a spectrum with $S^1$-action together with  $S^1$-equivariant maps $\varphi_p: X \ra X^{tC_p}$ for every prime $p$.
\end{defn}

Our main examples of cyclotomic spectra will come from topological Hochschild homology.
We will only define the cyclotomic structure on topological Hochschild homology in the setting of $\Einf$-ring spectra. (can also be constructed in greater generality for $\Eone$ but more complicated, see \cite[Chapter~3.1]{NS})
\begin{defn}
Let $R$ be an $\Eone$-ring spectrum.
We define the topological Hochschild homology of $R$ as follows
\begin{equation*}
    \THH(R)=R\otimes_{R\otimes R^{op}}R
\end{equation*}
\end{defn}
In the case of an $\Einf$-ring spectrum we can drop the $(\:)^{op}$. \\
The two main ingredients in the construction of the cyclotomic structure are the Tate diagonal and the following universal description of topological Hochschild homology. Note for this, that by definition $\THH(R)$ is an $R$-module and hence we have a map $R\ra\THH(R)$.

\begin{prop}[McClure-Schwänzl-Vogt]
Let R be an $\Einf$-ring spectrum, then the map $R\ra \THH(R)$ is initial as a map of \Einf-ring spectra from $R$ to an $\Einf$-ring spectrum with $S^1$-action.
\end{prop}
For the construction of the cyclotomic structure we first note that we can also construct an initial $\Einf$-map out of $R$, where the target is a spectrum with $C_p$-action: $R\ra \underbrace{R\otimes\dots\otimes R}_{p}$.
Since $\THH(R)$ has a $S^1$ and hence also a $C_p$-action we therefore get a map $R\otimes\dots\otimes R\ra \THH(R)$. Applying the $C_p$-Tate construction to this map and precomposing it with the Tate diagonal of $R$ we get an $\Einf$-map $R\xra{\Delta_p} R\otimes\dots\otimes R^{tC_p}\ra \THH(R)^{tC_p}$ whose target has $S^1/C_p\cong S^1$-action.
\\
Hence by the Proposition this factors through $\THH(R)$ and thus produces a map $\THH(R)\xra{\varphi_p} \THH(R)^{tC_p}$.
\[\begin{tikzcd}
	{R} & {\text{THH}(R)} \\
	{(R\otimes\dots\otimes R)^{tC_p}} & {\text{THH}^{tC_p}}
	\arrow[from=1-1, to=1-2]
	\arrow["{\varphi_p}", from=1-2, to=2-2, dashed]
	\arrow["{\Delta_p}"', from=1-1, to=2-1]
	\arrow[from=2-1, to=2-2]
\end{tikzcd}\]
This map equips the topological Hochschild homology of any $\Einf$-ring spectrum with a cyclotomic structure. 