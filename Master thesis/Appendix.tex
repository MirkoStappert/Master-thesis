\appendix
\chapter{K-Theory}
In this appendix, we give a brief review of the definition of algebraic K-theory for ring spectra. The construction can be performed as the composition of four intermediate steps, which we will outline now. \\
For this, let $R$ be a connective ring spectrum. \\
1) An $R$-module spectrum $M$ is called \textbf{finitely generated projective}, if it is a retract of a finitely generated, free $R$-module, i.e. there is another $R$-module $N$ and  $n\in\N$ such that $M\oplus N=R^n$.
Denote by $\Proj_R\subset \Mod_R$ the (full) subcategory of finitely generated projective $R$-module spectra. This is a symmetric monoidal $\infty$-category under the direct sum of $R$-modules. \\
2) For any $\infty$-category $\cC$, we can take its \textbf{groupoid core} $\cC^{\simeq}$, which is the largest subcategory of $\cC$ containing only equivalences. 
This is by definition an $\infty$-groupoid or equivalently a space. 
If $\cC$ has a symmetric monoidal structure, it induces\footnote{This is because the functor $(\phantom{\cC})^{\simeq}:\Cat_{\infty}\ra \Spc$ is symmetric monoidal, where both domain and target are symmetric monoidal under product. 
Commutative algebras in $\Cat_\infty$ are exactly symmetric monoidal categories.} an $\Einf$-structure on $\cC^{\simeq}$. 
In particular, $(\Proj_R)^{\simeq}$ is an $\Einf$-space.\\
3) Next we need the concept of an \textbf{grouplike $\Einf$-space}, which is simply an $\Einf$-space $X$, such that $\pi_0(X)$ is a group under the monoid operation induced by the $\Einf$-structure.
We define the category of grouplike $\Einf$-spaces $\CAlg_{\gp}(\Spc)$ as the full subcategory of $\CAlg(\Spc)$ containing those  $\Einf$-spaces, that are grouplike.
The crucial fact is that the thus provided inclusion $\CAlg_{\gp}(\Spc)\hookrightarrow \CAlg(\Spc)$ admits a left adjoint $X\to X^{\gp}$, called \textbf{group completion}. On the $\pi_0$ level, it actually induces the ordinary group completion which provides an initial abelian group for any commutative monoid. The existence of the adjoint can either be proved using the adjoint functor theorem or by providing a concrete construction, like $X^{\gp}=\Omega BX$. \\
4) The last step entails the equivalence of $\infty$-categories between connective spectra and grouplike $\Einf$-spaces: $\Sp_{\geq 0}\simeq \CAlg_{\gp}(\Spc)$. One functor is (on objects) given by $\Sp_{\geq 0}\ni X\mapsto \Omega^{\infty}X$. Let us denote its inverse by $B^\infty:\CAlg_{\gp}(\Spc)\rai \Sp_{\geq 0} $.  \\
These four construction now allow us to define the K-theory spectrum $K(R)$ of a connective ring spectrum $R$ as the composition of all of them. I.e. the \textbf{K-theory spectrum} of $R$ is the spectrum corresponding to the group completion of the groupoid core of the symmetric monoidal $\infty$-category of projective $R$-modules. 
\begin{align*}
        K: \Alg(\Sp)_{\geq 0}\ni R\mapsto \Proj_R\mapsto \Proj_R^{\simeq}\mapsto (\Proj_R^{\simeq})^{\gp}\mapsto B^{\infty}(\Proj_R^{\simeq})^{\gp}\in \Sp_{\geq 0}
\end{align*}
\begin{rmk}
    If $R$ is a commutative ring spectrum, $K(R)$ also is, more precisely for an $\En$-ring spectrum $R$, $K(R)$ is a connective $\mathbb{E}_{n-1}$-ring spectrum. In the language, used here, this is proved in \cite{GGNuniversalityloop}. The last chapter in there also contains more details on the above construction of the K-theory spectrum and proofs of the claimed facts.
\end{rmk}
\begin{rmk}
    In the case of a non-connective ring spectrum $R$, this definition is no longer suitable and we need to proceed differently. We should then not consider $\Proj_R$ but $\Perf_R$ instead. This is the smallest stable subcategory of $\Mod_R$ containing $R$, which is closed under retracts. The $\infty$-category $\Perf_R$ has the structure of a Waldhausen $\infty$-category in the sense of Barwick and we can define its K-theory, see \cite[Chapter~11]{barwickK-theory}. This construction is also suitably monoidal by \cite[Proposition~3.8]{barwick2013multiplicative}. In particular we again have the statement that $K(R)$ is an $\mathbb{E}_{n-1}$-ring spectrum for an $\En$-ring spectrum $R$, 
\end{rmk}
