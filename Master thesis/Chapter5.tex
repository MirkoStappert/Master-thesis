\chapter{Topological Hochschild Homology of CDVRs}\label{THHofCDVR}
In this chapter, we calculate the topological Hochschild homology of (certain) complete discrete valuation rings. \\
Our main input for this is the following fundamental result of Bökstedt.
% We give two statements.
\begin{thm}[Bökstedt] \label{Bökstedt}
On homotopy groups, there is an isomorphism of graded rings $\THH_*(\F_p)=\F_p[x]$  for  $x\in \THH_2(\F_p)$ a generator. 
We can also phrase this more conceptually as follows: $\THH(\F_p)$ is the free $\Eone$-$\F_p$-algebra on an element in degree 2, i.e. on the two sphere $S^2$. In other words $\THH(\F_p)\cong \F_p\otimes_{\S}\sip \Omega S^3$.
\end{thm}
Note that the second statement implies the first\footnote{Both statements are actually equivalent. For the other direction, observe that a generator of $\THH_2(\F_p)$ gives a map of $\Eone$-$\F_p$-algebras $\F_p\otimes_{\S}\sip \Omega S^3\ra \THH(\F_p)$ because the domain is the free $\Eone$-$\F_p$-algebra on an element in degree 2 and the target is even an $\Einf$-$\F_p$-algebra. By assumption, this map induces an isomorphism on homotopy groups and since equivalences of $\En$-algebras can be detected on the underlying spectra, the map is an equivalence of $\Eone$-$\F_p$-algebras.}, because the homotopy groups of $\F_p\otimes_{\S}\sip \Omega S^3$ are the $\F_p$-homology groups of $\Omega S^3$, which can be computed by the Serre spectral sequence associated to the path space fibration of $S^3$.
The ring structure on $H_*(\Omega S^3;\F_p)$ is given by the Pontryagin product induced by the $\Eone$-structure on $\Omega S^3$.
\newline
This periodicity in homotopy groups is called \textit{Bökstedt periodicity}.
In fact, the same phenomenon occurs in a much greater class of examples, namely for perfect $\F_p$-algebras and even all perfectoid rings (\cite[Chapter~6]{BMS2}). 
There is also a relative version for complete discrete valuation rings with perfect residue field of characteristic $p$.
The proof for these results bootstraps the Bökstedt periodicity of $\F_p$ using two base change formulas that we will introduce now. 

\section{Base change formulas}\label{basechangeformulas}
\begin{prop}\label{base change THH}
Let $k'\ra k$ be a map of $\Einf$-ring spectra and $R$ an \Eone-$k'$-algebra. Then we have 
\begin{equation*}
\THH(R/k')\otimes_{k'} k=\THH(R\otimes_{k'} k/k).
\end{equation*}
\end{prop}
\begin{proof}
The base change functor $-\otimes_{k'} k:\Mod_{k'}\ra \Mod_{k}$ is left adjoint to the forgetful functor, so it preserves all colimits. 
It furthermore carries a symmetric monoidal structure. 
Since topological Hochschild homology is built via the geometric realization of the cyclic bar construction it only uses the monoidal structure and geometric realization.
Hence any symmetric monoidal functor that also preserves geometric realizations will induce an equivalence on topological Hochschild homology. 
This applies in particular to the functor $-\otimes_{k'} k:\Mod_k'\ra \Mod_{k}$, since it preserves all colimits.
\end{proof}
\noindent An important special case is $k'=\S$, which gives us
$\THH(R)\otimes_\S k=\THH(R\otimes_\S k/k)$. \\
The next base change formula will allow us to `decompose' relative $\THH$ computations. We prepare it with a small Lemma.
\begin{lem}\label{tpovertp}
Let $k'\ra k$ be a map of $\Einf$-ring spectra and $M,N\in \Mod_k$ modules over $k$ (and thus also over $k'$). Consider $k$ as a module over $k\otimes_{k'} k$ via multiplication. We then have the following natural equivalence
\[
M\otimes_k N\simeq (M\otimes_{k'} N)\otimes_{k\otimes_{k'} k}k.
\]
\end{lem}
\begin{proof}
We have a canonical map of $k$-module spectra $M\otimes_{k'} N\ra M\otimes_k N$ induced by the map ${k'} \ra k$. Under the adjunction $-\otimes_{k\otimes_{k'} k}k:\Mod_{k\otimes_{k'} k}\rightleftarrows{} \Mod_k:\text{Forget}$ this is adjoint to the map 
\[
(M\otimes_{k'} N)\otimes_{k\otimes_{k'} k}k\rightarrow M\otimes_k N.
\]
We want to show that this is an equivalence. Because $\Mod_k$ is under colimits generated by $k$ and $-\otimes_{k'}-$ preserves colimits in both variables separately, it suffices to check the statement in the case $M=N=k$. There we have 
\[
(k\otimes_{k'} k)\otimes_{k\otimes_{k'} k}k\rightarrow k\otimes_k k,
\]
which is clearly an equivalence.
\end{proof}
\noindent Using exactly the same argument, we also obtain the result 
\[M_1\otimes_k M_2\otimes_k \dots \otimes_k M_n\lai (M_1\otimes_{k'} M_2\otimes_{k'} \dots \otimes_{k'} M_n)\otimes_{k\otimes_{k'}\dots \otimes_{k'} k }k,
\]
for $M_1,\dots M_n$ modules over $k$.
\begin{prop} \label{relative THH formula}
Let again $k'\ra k$ be a map of $\Einf$-ring spectra and $R$ an \Eone-$k$-algebra.
Then $k$ is a $\THH(k/k')$-module and we have
\begin{equation*}
    \THH(R/k)=\THH(R/k')\otimes_{\THH(k/k')} k
\end{equation*}
\end{prop}
\begin{proof}
Let us start by writing the left hand side in terms of its cyclic bar construction and applying the previous Lemma in the case where all $k$-modules are given by $R$.
%Observation: $R\otimes_k R\simeq (R\otimes_\S R)\otimes_{k\otimes_\S k}k$ and also \\ $R\otimes_k\dots \otimes_k R\simeq (R\otimes_\S\dots\otimes_\S R)\otimes_{k\otimes_\S\dots \otimes_\S k}k$ \\ This allows us to rewrite:
\begin{align*}
  \THH(R/k)&=\colim_{\Delta^{\op}}\left(\dots \rightthreearrow R\otimes_k R\rightrightarrows R\right) \\
  &\stackrel{\ref{relative THH formula}}{=}\colim_{\Delta^{\op}}\left(\dots \rightthreearrow (R\otimes_{k'} R)\otimes_{k\otimes_{k'} k}k\rightrightarrows R\right) \\
  &=\colim_{\Delta^{\op}}\left(
  (\dots  \rightthreearrow R\otimes_{k'} R\rightrightarrows R)
  \otimes_{(\dots  k\otimes_{k'} k \rightrightarrows k)} (\dots \rightthreearrow k \rightrightarrows k)\right)
\end{align*}
In the last line we consider the category $\Fun(\Delta^{\op},\Sp)$ as a symmetric monoidal category equipped with the Day convolution tensor product (see \cite[Section~2.2.6]{lurie2017higher}). In this situation $(\dots \rightrightarrows  k\otimes_{k'} k \rightrightarrows k)$ is a commutative algebra object in $\Fun(\Delta^{\op},\Sp)$ and $(\dots  \rightthreearrow R\otimes_{k'} R\rightrightarrows R)$, $(\dots \rightthreearrow k \rightrightarrows k)$   are modules over it. The tensor product over
$(\dots \rightrightarrows  k\otimes_{k'} k \rightrightarrows k)$ is then to be understood in the sense of \cite[Section~4.5.2]{lurie2017higher}. \\ 
Now we can use that $\Delta^{\op}$ is sifted, i.e. the diagonal $\Delta^{\op}\ra \Delta^{\op}\times \Delta^{\op}$ is cofinal and thus also $\Delta^{\op}\ra \Delta^{\op}\times \Delta^{\op}\times \Delta^{\op}$ is cofinal (see \cite[Lemma~5.5.8.4]{HigherToposTheory}). 
Applying this cofinality (under the equivalent characterization of cofinality proven in \cite[Proposition~4.1.1.8]{HigherToposTheory}); that the tensor product commutes with colimits in both variables separately and that we can also pull geometric realizations out of the ring, we are tensoring over\footnote{This itself uses the construction of the relative tensor product as the geometric realization of the bar construction (\cite[Section~4.4.2]{lurie2017higher}) and once more the cofinality of $\Delta^{\op}\ra \Delta^{\op}\times \Delta^{\op}$.},  we obtain:
\begin{align*}
    \THH(R/k)&=\colim_{\Delta^{\op}\times \Delta^{\op}\times \Delta^{\op}}\left(
  (\dots  \rightthreearrow R\otimes_{k'} R\rightrightarrows R)
  \otimes_{(\dots  k\otimes_{k'} k \rightrightarrows k)} (\dots \rightthreearrow k \rightrightarrows k)\right) \\
  &=\colim_{\Delta^{\op}}(\dots  \rightthreearrow R\otimes_{k'} R\rightrightarrows R)
  \otimes_{
  \colim_{\Delta^{\op}}
  (\dots  k\otimes_{k'} k \rightrightarrows k)
  }
  \colim_{\Delta^{\op}}(\dots \rightthreearrow k \rightrightarrows k) \\
  &=\THH(R/k)\otimes_{\THH(k/k')}k 
\end{align*} \qedhere
\end{proof}
\noindent
Again there is the important special case of $k'=\S$, which gives us $\THH(R/k)=\THH(R)\otimes_{\THH(k)} k$. \newline
Note also that Proposition \ref{relative THH formula} implies Proposition \ref{base change THH}.
Indeed 
\begin{align*}
    \THH(R\otimes_{k'} k/k) &\stackrel{\ref{relative THH formula}}{=}
    \THH(R\otimes_{k'}k /k')\otimes_{\THH(k/k')} k \\
    &=\THH(R/k')\otimes_{k'}\THH(k/k')\otimes_{\THH(k/k')} k \\
    &=\THH(R/k')\otimes_{k'}k.
\end{align*}



%%%%%%%%%%%%%%%%%%%%%%%%%%%%%%%%%%%%%%%%%
\section{Bökstedt periodicity for perfect $\F_p$-algebras}
%%%%%%%%%%%%%%%%%%%%%%%%%%%%%%%%%%%%%%%%%
In this section, we prove Bökstedt periodicity for perfect\footnote{For every $\F_p$-algebra $R$ we have that the Frobenius $\varphi:R\ra R, \ x\mapsto x^p$ is a ring homomorphism. We call $R$ perfect if $\varphi$ is an isomorphism.} $\F_p$-algebras.
Examples include perfect fields, in particular finite fields $\F_{p^n}$, as well as non-Noetherian examples like $\F_p[x^{1/p^\infty}]$.
For the proof, we will rely on  the following construction that establishes that we can not only deform perfect $\F_p$-algebras to characteristic zero via Witt vectors, but even to the sphere spectrum using a spherical analogue of the Witt vectors.
(see \cite[Example~5.2.7]{lurie2018elliptic})
\begin{prop}
Let $k$ be a perfect $\F_p$-algebra. Then there exists a $p$-complete $\Einf$-ring spectrum $\S_{W(k)}$ - the \textbf{spherical Witt vectors} of $k$ - with base change $\S_{W(k)}\otimes_\S \F_p\simeq k $.
We thus get the following diagram of deformations of $k$:
% https://q.uiver.app/?q=WzAsNixbMCwwLCJcXG1hdGhiYntTfV97VyhrKX0iXSxbMSwwLCJXKGspIl0sWzIsMCwiayJdLFsyLDEsIlxcbWF0aGJie0Z9X3AiXSxbMSwxLCJcXG1hdGhiYntafSJdLFswLDEsIlxcbWF0aGJie1N9Il0sWzAsMV0sWzEsMl0sWzMsMl0sWzQsMV0sWzQsM10sWzUsMF0sWzUsNF1d
\[\begin{tikzcd}
	{\mathbb{S}_{W(k)}} & {W(k)} & k \\
	{\mathbb{S}} & {\mathbb{Z}} & {\mathbb{F}_p}
	\arrow[from=1-1, to=1-2]
	\arrow[from=1-2, to=1-3]
	\arrow[from=2-3, to=1-3]
	\arrow[from=2-2, to=1-2]
	\arrow[from=2-2, to=2-3]
	\arrow[from=2-1, to=1-1]
	\arrow[from=2-1, to=2-2]
\end{tikzcd}\]
\end{prop}
In general for any $\F_p$-algebra $k$ we have a map $\F_p\ra k$, which induces a map of $\Einf$-ring spectra $\THH(\F_p)\ra \THH(k)$. 
Since the target is a $k$-module spectrum, this further refines to $\THH(\F_p)\otimes_{\F_p} k\ra \THH(k)$.
The statement of Bökstedt periodicity for perfect $\F_p$-algebras is now the following.
\begin{thm} \label{Bökstedt for perfect rings}
For a perfect $\F_p$-algebra $k$ the above map $\THH(\F_p)\otimes_{\F_p} k  \ra \nolinebreak \THH(k)$ is an equivalence of $\Einf$-$k$-algebras. This in particular determines the homotopy groups as a graded ring
\[\THH_*(k)=k[x], \ |x|=2.
\]
\end{thm}
\begin{proof}
Using the previous Proposition and that $\THH$ is symmetric monoidal, we see that:
\begin{align*}
    \THH(k)&=\THH(\F_p\otimes_\S \S_{W(k)}) \\
           &=\THH(\F_p)\otimes_\S \THH(\S_{W(k)}) \\
           &=\THH(\F_p)\otimes_{\F_p}(\F_p\otimes_\S \THH (\S_{W(k)}))
\end{align*}
By Proposition \ref{base change THH} the term in brackets can be identified as \\ $\F_p\otimes_\S \THH (\S_{W(k)})=\THH(\S_{W(k)}\otimes_\S \F_p/\F_p)=\THH(k/\F_p)$. 
Furthermore since $\F_p$ is an ordinary ring $\THH(k/\F_p)=\HH(k/\F_p)$. 
To prove the desired result, it remains to show that $\HH(k/\F_p)$ is $k$ concentrated in degree 0. 
Because $k$ is commutative $\HH_0(k/\F_p)=k$, so we need to prove the vanishing of the higher homology groups. \newline
We claim that for a perfect $\F_p$-algebra $R$ the cotangent complex vanishes $\bL_{R/\F_p}=0$.
Together with the Hochschild-Kostant-Rosenberg filtration on Hochschild homology, this gives that $\HH_i(k/\F_p)=0$ for $i>0$, which then finishes the proof. For the vanishing of the cotangent complex, note that for any $\F_p$ algebra the Frobenius induces multiplication by $p$ on the cotangent complex\footnote{To see this, recall that the cotangent complex is by definition the non-abelian derived functor of Kähler differentials. To compute $\bL_{R/\F_p}$ we thus have to simplicially resolve $R$ by polynomial rings over $\F_p$ and then apply Kähler differentials level-wise. The Frobenius then acts via $dx\mapsto d(x^p)=pdx$.}, i.e. the zero map. But since $R$ is perfect, the Frobenius $\varphi:R\ra R$ is an isomorphism and by functoriality still an isomorphism on the cotangent complex. The only way for the zero map to be an isomorphism is if already $\bL_{R/\F_p}=0$.
\end{proof}



\section{Relative Bökstedt periodicity for CDVRs}
Let $R$ be a discrete valuation ring, i.e. a local principal ideal domain that is not a field. Denote its unique maximal ideal by $\m$ and the residue field by $k=R/\m$.
In this section, we assume that $R$ is complete with respect to the ideal $\m$, i.e. $R\rai \lim (R/\m\la R/\m^2 \la \dots)$ and that the residue field $k$ is perfect of characteristic $p>0$. We will abbreviate the term complete discrete valuation ring to CDVR.
Any generator $\pi\in R$ of the maximal ideal will be called a \textit{uniformizer}. Completeness with respect to the ideal $\m$ is then equivalent to completeness with respect to the element $\pi$.
\newline
Either $R$ and $k$ are both of characteristic $p$ (then we say that $R$ is of \textit{equal characteristic}) or $R$ is of characteristic 0 and $k$ is of characteristic $p$ (the \textit{mixed characteristic} case).
In the equal characteristic setting, it is known that $R=\nolinebreak k[[x]]$ for $k$ a perfect field of characteristic $p$. 
In the mixed characteristic case there are more examples: We again get an example for every perfect field of characteristic $p$, namely $W(k)$. 
But there are also ramified examples like $\Z_p[\sqrt[n]{p}]$. or $\Z_p[\zeta_p]$. We can fully characterize them, as certain extensions of the unramified examples. This will be extremely useful later on.
\begin{prop} \label{description mixed char CDVR}   
    Let $R$ be a complete discrete valuation ring of mixed characteristic with perfect residue field $k$. Then $R$ is of the form $R\cong W(k)[z]/E(z)$ for an Eisenstein polynomial\footnote{A polynomial is called Eisenstein if it is of the form $a_nz^n+a_{n-1}z^{n-1}+\dots a_1z+a_0$ such that for $i=0,\dots n-1$ we have $p|a_i$, $p$ does not divide $a_n$ and $p^2$ does not divide $a_0$.} $E(z)\in W(k)[z]$. Under the map $W(k)[z]\ra R$, the element $z$ goes to a uniformizer in $R$ and $E$ is a minimal polynomial of $\pi$.
\end{prop}
\begin{proof}
    See \cite[Section 2.5]{SerreLocal}.
\end{proof}

To calculate $\THH(R)$ we first work relative to the maximal ideal which allows us to use the result from the previous section because the residue field is by assumption perfect. In the next section, we will then obtain the non-absolute result using a descent spectral sequence. \newline
We will work relative to the spectrum $\S[z]\coloneqq \sip (\N_0)$, the \textbf{spherical monoid ring} of $\N_0$. Since $\N_0$ is a commutative monoid, hence an $\Einf$-algebra in spaces and $\sip:\Spc\ra \Sp$ has a strong symmetric monoidal structure, we get that $\S[z]$ is an $\Einf$-ring spectrum.
To work relative to $\S[z]$, we need an $\Eone$-map $\S[z]\ra HR$, which informally is just given by $z\mapsto \pi$ for $\pi$ a uniformizer in $R$. 
To make this map precise, note that the ($\sip, \Omega^\infty$)-adjunction is compatible with the symmetric monoidal structures\footnote{i.e. $\sip$ is (strong) symmetric monoidal and $\Omega^\infty$ is (lax) symmetric monoidal with respect to the cartesian product on $\Spc$ and the tensor product on $\Sp$.} on $\Spc$ and $\Sp$, so we get an induced
%\todo{reference: Vielleicht GGN 3.6}
adjunction between $\CAlg(\Spc)$ and $\CAlg(\Sp)$. 
Thus giving a map $\S[z]\ra   R$  of $\Einf$-ring spectra is equivalent to giving a map $\N_0\ra R$ of $\Einf$-algebras in spaces.
This is just a morphism of ordinary commutative monoids and since $\N_0$ is the free monoid (in sets) on one generator, we get such a map for each element of $R$. In particular we have a map corresponding to $\pi\in R$ and this is what we mean by $\S[z]\ra R,\ z\mapsto \pi$.
We are now ready to state and prove a relative variant of Bökstedt's theorem for CDVRs.
\begin{thm} \label{Relative Bökstedt for CDVR}
Let $R$ be a complete discrete valuation ring with perfect residue field of characteristic p. Choose a uniformizer $\pi\in R$ and give $R$ a $\S[z]$-algebra structure by $z\mapsto \pi$. Then
\begin{equation*}
    \THH_*(R/\S[z];\Z_p)=R[x], \ |x|=2.
\end{equation*}
\end{thm}
\begin{proof}
Let us first treat the case that $R$ is mixed characteristic. Since $R$ is characteristic 0, $p$ is not zero in $R$, so $p=u\pi^n$ for $u$ a unit in $R$ and $n\geq 1$. This implies that an $R$-module is derived $p$-complete if and only if it is derived $\pi$-complete by  \cite[\href{https://stacks.math.columbia.edu/tag/091Q}{Tag 091Q}]{stacks-project}.
In particular we get that the homotopy groups of  $\THH(R/\S[z];\Z_p)$ are derived $\pi$-complete because they are by definition derived $p$-complete.
Using Proposition \ref{base change THH} and that the tensor product preserves colimits in both variables we can now calculate
\begin{align*}
    \THH(R/\S[z];\Z_p)\otimes_{\S[z]} \S&\stackrel{\ref{base change THH}}{=} \THH(R\otimes_{\S[z]}\S/ \S;\Z_p) 
    \\
    &=\THH(R\otimes_{\S[z]}\cofib\left(\S[z]\xrightarrow{\cdot z} \S[z]\right);\Z_p) \\
    &=\THH(R/\pi;\Z_p)=\THH(k;\Z_p)=\THH(k).
\end{align*}
In the last line we used that $z$ acts by multiplication by $\pi$ on $R$ and that $\THH(k)$ is already $p$-complete since all its homotopy groups are already derived $p$-complete (even classically $p$-complete). On the other hand, again using that the tensor product preserves colimits, we can also see that $\THH(R/\S[z];\Z_p)\otimes_{\S[z]} \S=\THH(R/\S[z];\Z_p)/\pi$. Hence we get the cofiber sequence 
\begin{equation*}
    \THH(R/\S[z];\Z_p)\xrightarrow{\cdot \pi}\THH(R/\S[z];\Z_p)\ra \THH(k).
\end{equation*}
Since $k$ is by assumption perfect of characteristic $p$, it is in particular a perfect $\F_p$-algebra, so Theorem \ref{Bökstedt for perfect rings} applies and tells us the homotopy groups of $\THH(k)$. Thus the long exact sequence for the odd homotopy groups looks as follows 
\[
\dots \ra k\ra \THH_{2i+1}(R/\S[z];\Z_p)\xrightarrow{\cdot \pi} \THH_{2i+1}(R/\S[z];\Z_p) \ra 0 \ra \dots
\]
This tells us that the odd homotopy groups of $\THH(R/\S[z];\Z_p)$ are 0 mod $\pi$.
But by our discussion at the beginning of the proof we also know that they are derived $\pi$-complete, so they must already be 0 before taking mod $\pi$ reduction (see \cite[\href{https://stacks.math.columbia.edu/tag/09B9}{Tag 09B9}]{stacks-project}).
\newline
For the identification of the (non-negative) even homotopy groups, we proceed similarly.
The long exact sequence takes the form
\[
\dots \ra 0\ra \THH_{2i}(R/\S[z];\Z_p)\xrightarrow{\cdot \pi} \THH_{2i}(R/\S[z];\Z_p) \ra k \ra \dots,
\]
which gives us that they are $\pi$-torsion free and their mod $\pi$ reduction is $k$. They are also derived $\pi$-complete.
The same holds for $R$, so once we have an $R$-module map (over $k$) we get an isomorphism by the derived Nakayama Lemma ($\pi$-torsion freeness guarantees that the derived mod $\pi$ reduction agrees with the ordinary one). 
Let $M\coloneqq \THH_2(R/\S[z];\Z_p)$. By the long exact sequence we have a surjective map $M\ra k$. Giving an $R$-module map $R\ra M$ is the same as choosing an element $x \in M$. 
To also ensure that the map is an equivalence after reducing mod $\pi$, the reduction of $x$ must generate $k$ as an $R$-module. So we have to choose a preimage of any non-zero element of $k$ under the map $M\ra k$, which always exists, since the map is surjective by the long exact sequence. 
Thus choosing such a lift provides an isomorphism $\THH_2(R/\S[z];\Z_p)\cong R$. The same applies to all positive even homotopy groups once we also choose lifts for all of them. A priori the choices of all these lifts might be unrelated. But since the map $\THH_*(R/\S[z];\Z_p)\ra \THH_*(k)$ is multiplicative, the elements $x^{i}\in \THH_{2i}(R/\S[z];\Z_p)$ provide canonical choices of all of those. 
Hence the choice of $x$ already suffices and gives us an isomorphism of graded rings $\THH_*(R/\S[z];\Z_p)\rai \THH_*(k)$.


If $R$ is of equal characteristic, we have already remarked that necessarily $R=k[[z]]$ for $k$ perfect of positive characteristic.
We proceed similarly to the proof of Theorem \ref{Bökstedt for perfect rings}. 
Defining $\S_{W(k)}[[z]]$ as the $z$-adic completion of $\S_{W(k)}\otimes_\S \S[z]$, gives us the following base change
\begin{equation*}
    \S_{W(k)}[[z]]\otimes_\S \F_p=k[[z]].  
\end{equation*} 
%Refer to Lemma \ref{PropertiesTHH} \todo{do that}\\
To see this, note that the right hand side is the $z$-adic completion of the left hand side, as in general we have $(X\otimes_\S Y)^\wedge_z=(X^\wedge_z\otimes_\S Y^\wedge_z)^\wedge_z$. To establish that the map to the completion is an isomorphism we thus have to see that the left hand side is already $z$-complete. This holds because $\F_p$ is finite type over the sphere and $ \S_{W(k)}[[z]]$ is $z$-complete. \\
% , we can write it as a filtered colimit of finite spectra along strictly increasingly connective maps, i.e. $\F_p=\colim_i X_i$ with $X_i$ finite and $X_i\rightarrow X_{i+1}$ is $i$-connective. Hence $\pi_n(\F_p\otimes_\S \S_{W(k)}[[z]])=\pi_n(X_n\otimes_\S \S_{W(k)}[[z]])$.
% As $X_n$ is finite and completion is preserved under \textit{finite} colimits, $X_n\otimes_\S \S_{W(k)}[[z]]$ is $z$-complete, which implies that its homotopy groups are derived $z$-complete. Thus all homotopy groups of $\S_{W(k)}[[z]]\otimes_\S \F_p$ are derived $z$-complete, meaning that $\S_{W(k)}[[z]]\otimes_\S \F_p$ is $z$-complete\footnote{\label{sequential p completion}Note that this shows in general that although being complete is normally not closed under infinite colimits it is indeed preserved if we are taking a sequential colimit along strictly increasingly connective maps. We will use this again later.}.
After establishing this base change, we can use  \ref{base change THH} again :
\begin{align*}
    \THH(k[[z]]/\S[z])&=\THH(\S_{W(k)}[[z]]\otimes_\S \F_p) \\
    &=\THH(\S_{W(k)}[[z]]/\S[z])\otimes_\S \THH(\F_p/\S[z]) \\
    &=\left(\THH(\S_{W(k)}[[z]]/\S[z])\otimes_\S \F_p\right)\otimes_{\F_p}\THH(\F_p) \\
    &\stackrel{\ref{base change THH}}{=}
    \THH(\S_{W(k)}[[z]]\otimes_\S \F_p/\S[z]\otimes_\S \F_p)
    \otimes_{\F_p}\THH(\F_p) \\
    &=\HH(k[[z]]/\F_p[z])\otimes_{\F_p}\THH(\F_p)
\end{align*}
So we have to show that $\HH_i(k[[z]]/\F_p[z])$ vanishes for $i>0$. For perfect $\F_p$-algebras we used perfectness to show this, which is not possible here, because $k[[z]]$ is clearly not perfect. But the map $\F_p[z]\ra k[[z]]$ is relatively perfect\footnote{The naming comes from the fact that an $\F_p$-algebra $R$ is perfect if and only if $\F_p\rightarrow R$ is relatively perfect}, which by definition means that the following square is a pushout in $\CAlg$.
% https://q.uiver.app/?q=WzAsNCxbMCwxLCJcXG1hdGhiYntGfV9wW3pdIl0sWzEsMSwia1tbel1dIl0sWzAsMCwiXFxtYXRoYmJ7Rn1fcFt6XSJdLFsxLDAsImtbW3pdXSJdLFsyLDAsIlxcdmFycGhpIiwyXSxbMCwxXSxbMiwzXSxbMywxLCJcXHZhcnBoaSIsMl0sWzIsMSwiIiwwLHsic3R5bGUiOnsibmFtZSI6ImNvcm5lci1pbnZlcnNlIn19XV0=
\[\begin{tikzcd}
	{\mathbb{F}_p[z]} & {k[[z]]} \\
	{\mathbb{F}_p[z]} & {k[[z]]}
	\arrow["\varphi"', from=1-1, to=2-1]
	\arrow[from=2-1, to=2-2]
	\arrow[from=1-1, to=1-2]
	\arrow["\varphi"', from=1-2, to=2-2]
	\arrow["\ulcorner"{anchor=center, pos=0.125}, draw=none, from=1-1, to=2-2]
\end{tikzcd}\]
To see this, we first observe that the vertical maps in the diagram are on the level of ordinary rings simply the inclusions $\F_p[z^p]\hookrightarrow\F_p[z]$ and $k[[z^p]]\hookrightarrow k[[z]]$ (where we implicitly identify $k$ with itself under the action of Frobenius using that $k$ is perfect). Thus both maps exhibit the target as a free module over the domain on the basis $\{1,z,z^2,\dots, z^p\}$. Because equivalences in $\CAlg$ are detected on the underlying spectra, this also means that
\begin{equation*}
    \F_p[z]\otimes_{\F_p[z^p]}k[[z^p]]\cong k[[z]]
\end{equation*}
which implies that the square is a pushout.
Using this isomorphism, we obtain an isomorphism in Hochschild homology as well:
\begin{align*}
    \HH(k[[z]]/\F_p[z])
    &\lai
    \HH((\F_p[z]\otimes_{\F_p[z^p]}k[[z^p]])/
    \F_p[z])\\
    &= 
    \HH(\F_p[z]/\F_p[z])\otimes_{\F_p[z^p]}
    \HH(k[[z^p]]/
    \F_p[z])\\
    &= \F_p[z]\otimes_{\F_p[z^p]}
    \HH(k[[z^p]]/
    \F_p[z])
\end{align*}
We now want to show that this map induces the zero map on all $\pi_n, \ n>0$ which will then finish the proof as it is also an isomorphism.
Let us first observe that since $\F_p[z]$ is free as an $\F_p[z^p]$-module we can pull the tensor product out of the homotopy/homology group:
\[
\pi_n(\F_p[z]\otimes_{\F_p[z^p]}
    \HH(k[[z^p]]/
    \F_p[z]))=\F_p[z]\otimes_{\F_p[z^p]}
    \HH_n(k[[z^p]]/
    \F_p[z])
\]
Under this identification the map in question
\[
\F_p[z]\otimes_{\F_p[z^p]}
    \HH_n(k[[z^p]]/
    \F_p[z])\rightarrow \HH_n(k[[z]]/\F_p[z])
\] 
is adjoined (under the left adjoint to the forgetful functor $\text{Mod}_{\F_p[z]}\ra \text{Mod}_{\F_p[z^p]}$) to the map, which is itself obtained by applying $\HH_n(-/\F_p[z])$ to the inclusion $k[[z^p]]\ra k[[z]]$.
As the Hochschild homology groups are the homotopy groups of a simplicial commutative algebra and the latter always carry a divided power structure (see eg. \cite[Section~4]{Richterdividedpower}, this inclusion  necessarily induces the zero map because we are in characteristic $p$.
\end{proof}



Using the same technique we can also compute relative $\THH$ of quotients of the CDVRs that we have considered so far. In any discrete valuation ring all ideals are powers of the maximal ideal. So concretely we are dealing with rings of the form $R/\pi^n$, where $R$ is a complete discrete valuation ring with perfect residue field of positive characteristic and $\pi$ a uniformizer. For instance this covers the cases $\Z_p/p^n=\Z/p^n$ or truncated polynomial algebras $k[[x]]/x^n=k[x]/x^n$ for $k$ perfect of positive characteristic.
\newline
For these quotients we do not quite get polynomial homotopy groups as before but we need an additional divided power generator compensating for the quotient.

\begin{prop}\label{relative Bökstedt for quotients}
Let $R$ be a complete discrete valuation ring with perfect residue field of positive characteristic, $n\geq 1$ and $R'=R/\pi^n$. Then we have:
\[ \THH_*(R'/\S[z])=R'[x]\langle y\rangle, \ |x|=|y|=2
\]

\end{prop}
%(We can drop $p$-completions compared to the previous results because our rings are $p$-power-torsion and hence all modules are already $p$-complete)

\begin{proof}
Since $\pi$ is not a zero divisor, the ordinary and derived quotient agree, so $R'=R\otimes_{\S[z]}(\S[z]/z^n) $.
Here we write $\S[z]/z^n=\Sigma^\infty M$, where $M$ is the (pointed) monoid $M=\{0,1,x,\dots, x^{n-1}\}$ with multiplicative monoid structure and $x^a\cdot x^b=0$ if $a+b\geq n$. By the strong symmetric monoidality of $\Sigma^\infty:(\mathcal{S}_*,\wedge)\rightarrow (\Sp,\otimes_\S)$ we obtain an $\Einf$-structure\footnote{The underlying spectrum of $\S[z]/z^n$ is simply given by taking the cofiber of $\S[z]\xrightarrow{\cdot z^n}\S[z]$ but this has a priori no ring structure anymore. That is why we give the alternative description which makes the existence of an $\Einf$-structure clear.} on $\S[z]/z^n$. \\
Using this identification of $R'$, we can calculate:
\begin{align*}
    \THH(R'/\S[z])&=\THH(R/\S[z])\otimes_{\S[z]}\THH(\S[z]/z^n/\S[z]) \\
    &=\THH(R/\S[z])\otimes_{\Z\otimes_\S \S[z]}\left(\Z\otimes_\S \THH((\S[z]/z^n)/\S[z]\right) \\
    &\stackrel{\ref{base change THH}}{=}
    \THH(R/\S[z])\otimes_{\Z[z]}\HH\left((\Z[z]/z^n)/\Z[z]\right) 
\end{align*}
As $p$ is nilpotent in $R'$, every $R'$-module spectrum is already $p$-complete, in particular $\THH(R'/\S[z])$ must be $p$-complete. This means that it does not matter whether we $p$-complete one (or even both) of the factors, i.e. we get 
\begin{equation*}
     \THH(R'/\S[z])=\THH(R/\S[z];\Z_p)\otimes_{\Z[z]}\HH\left((\Z[z]/z^n)/\Z[z]\right)
\end{equation*}
We completely understand the homotopy groups of the first factor by the previous theorem, so again we need to do a calculation in Hochschild homology. The derived tensor product $(\Z[z]/z^n)\otimes_{\Z[z]}^L(\Z[z]/z^n)=\Lambda_{\Z[z]/z^n}(e)$ is given by an exterior algebra over $\Z[z]/z^n$ on a generator $e$ in degree 1. Thus Hochschild homology is given by 
\[
\HH\left((\Z[z]/z^n)/\Z[z]\right)=
(\Z[z]/z^n)\otimes_{\Lambda_{\Z[z]/z^n}(e)}^L (\Z[z]/z^n).
\]
To calculate this derived tensor product, we can explicitely resolve $\Z[z]/z^n$ as a free divided power algebra over $\Lambda_{\Z[z]/z^n}(e)$ as follows
\[ 
\Lambda_{\Z[z]/z^n}(e)\langle y \rangle \coloneqq \frac{\Lambda_{\Z[z]/z^n}(e)[y_1,y_2,\dots]}{y_iy_j=\binom{i+j}{i}y_{i+j}}, \ \partial y_i=ey_{i-1}, \ |y_i|=2i.
\]
Hence we get the following result for Hochschild homology
\[
\HH\left((\Z[z]/z^n)/\Z[z]\right)=
\Lambda_{\Z[z]/z^n}(e)\langle y \rangle
\otimes_{\Lambda_{\Z[z]/z^n}(e)} (\Z[z]/z^n)= (\Z[z]/z^n)\langle y\rangle.
\]
This finally enables us to compute the  homotopy groups:

\begin{equation*}
    \THH_*(R'/\S[z])=R'[x]\langle y\rangle, \ |x|=|y|=2.
\qedhere
\end{equation*} 
\end{proof}



\section{Absolute THH of CDVRs}\label{AbsoluteTHHofCDVRs}
In this section, we calculate the absolute $\THH$ of complete discrete valuation rings with perfect residue field of positive characteristic. We have already computed it relative to $\S[z]$. To get from there to the absolute case we use a descent spectral sequence that we construct now. 
\begin{constr} \label{descentSS}
Let $R$ be an ordinary commutative ring equipped with a map of rings $\Z[z]\ra R$ (i.e. the choice of an element of $R$). This also gives $R$ the structure of an $\Einf$-$\S[z]$-algebra, via $\S[z]\ra \Z[z]$. We can filter $\HH(\Z[z]$ by its (very short) Whitehead tower:

\[ 0=\tau_{\geq2} \HH(\Z[z])\ra  \tau_{\geq1} \HH(\Z[z])\ra \tau_{\geq0} \HH(\Z[z])=\HH(\Z[z])
\]
Observe that $\HH(\Z[z])\stackrel{\ref{base change THH}}{=}\THH(\S[z])\otimes_\S \Z$, which gives us an $\Einf$-$\HH(\Z[z])$-algebra structure on $\THH(R)$ via the map $\THH(\S[z])\ra \THH(R)$. We thus have the strong monoidal base change functor $\Mod_{\HH(\Z[z])}\ra \Mod_{\THH(R)}$ and can level-wise apply it to the filtration to get a multiplicative filtration on $\THH(R)$
\[ 0 \ra  \tau_{\geq1} \HH(\Z[z])\otimes_{\HH(\Z[z])} \THH(R)
\ra \HH(\Z[z])\otimes _{\HH(\Z[z])} \THH(R)=\THH(R)
\]
This filtered spectrum gives rise to a spectral sequence, whose $E^2$-page is given by the homotopy groups of the associated graded $E^2_{i,j}=\pi_{i+j} \gr^j$, see\footnote{Note that Lurie works with ascending filtrations, while the Whitehead filtration is descending. The differentials and indexing therefore work differently, which is why we reindex to be consistent in our usage of homological Serre grading. As a result of this reindexing, the differentials work in such a way that we start with the $E^2$-page instead of an $E^1$-page.} e.g. \cite[Chapter~1.2.2]{lurie2017higher}.
To compute the $E^2$-page, we thus have to identify the associated graded. 
\newline
Since tensor products preserve cofiber sequences, the associated graded of the filtration on $\THH(R)$ can simply be obtained by tensoring the associated graded of the filtration on $\HH(\Z[z])$ with $\THH(R)$. The associated graded of the filtration on $\HH(\Z[z])$ is given by $\HH_i(\Z[z])=\Omega_{\Z[z]/\Z}^i$.
Using our base change formulas, we can relate this to relative $\THH$:
\begin{align*}
    \gr^j&=\THH(R)\otimes_{\HH(\Z[z])}\Omega_{\Z[z]/\Z}^j[j]
    \\
    &=\THH(R)\otimes_{\HH(\Z[z])}\Z[z]\otimes_{\Z[z]}\Omega_{\Z[z]/\Z}^j[j]
    \\
    &\stackrel{\ref{base change THH}}{=}\left(\THH(R)\otimes_{\THH(\S[z])\otimes_\S \Z} (\S[z]\otimes_\S \Z)\right)\otimes_{\Z[z]}\Omega_{\Z[z]/\Z}^j [j]
    \\
    &=\left(\THH(R)\otimes_{\THH(\S[z])}\S[z]\right) 
    \otimes_{\Z[z]}\Omega_{\Z[z]/\Z}^j[j] 
    \\
    &\stackrel{\ref{relative THH formula}}{=}
    \THH(R/\S[z])\otimes_{\Z[z]}\Omega_{\Z[z]/\Z}^j[j]
\end{align*}
Therefore we get a homological (Serre graded) spectral sequence
\[
E_{i,j}^2=\THH_i(R/\S[z])\otimes_{\Z[z]}\Omega_{\Z[z]/\Z}^j \Rightarrow \THH_{i+j}(R).
\]
The spectral sequence is concentrated in degrees $(i,j)\in [0,\infty)\times [0,1]$, in particular it is first quadrant and converges. Furthermore the spectral sequence is multiplicative since the Whitehead filtration is multiplicative\footnote{i.e. it is an algebra in the $\infty$-category of filtered spectra $\Fun((N\Z)^{\op},\Sp)$ equipped with the Day convolution tensor product}.
\newline
By $p$-completing everywhere and using that $p$-completion is exact we also get the following spectral sequence with the same formal properties:
\[
E_{i,j}^2=\THH_i(R/\S[z];\Z_p)\otimes_{\Z_p[z]}\Omega_{\Z_p[z]/\Z_p}^j \Rightarrow \THH_{i+j}(R;\Z_p)
\]

\end{constr}
In our case of interest $R$ is again a CDVR with perfect residue field of positive characteristic. We completely know the $E^2$-page which by Theorem \ref{Relative Bökstedt for CDVR} takes the form 
$E^2=R[x]\otimes_{\Z_p}\Lambda_{\Z_p}(dz)$ for elements $|x|=(0,2)$ and $|dz|=(1,0)$ as pictured below.% https://q.uiver.app/?q=WzAsMjgsWzEsMywiUiJdLFsyLDMsIjAiXSxbMiwyLCIwIl0sWzMsMywiUlxce3hcXH0iXSxbMSwyLCJSXFx7IGR6XFx9Il0sWzMsMiwiUlxce3hkelxcfSJdLFs0LDIsIjAiXSxbNCwzLCIwIl0sWzEsMSwiMCJdLFsyLDEsIjAiXSxbMywxLCIwIl0sWzQsMSwiMCJdLFs1LDMsIlJcXHt4XjJcXH0iXSxbNSwyLCJSXFx7eF4yZHpcXH0iXSxbNSwxLCIwIl0sWzYsMywiXFxkb3RzIl0sWzYsMiwiXFxkb3RzIl0sWzYsMSwiXFxkb3RzIl0sWzEsMCwiXFxkb3RzIl0sWzIsMCwiXFxkb3RzIl0sWzMsMCwiXFxkb3RzIl0sWzQsMCwiXFxkb3RzIl0sWzUsMCwiXFxkb3RzIl0sWzYsMCwiXFxkb3RzIl0sWzAsNF0sWzAsM10sWzYsNF0sWzAsMF0sWzMsNF0sWzEyLDVdLFsyNCwyNiwiIiwwLHsib2Zmc2V0IjotNH1dLFsyNCwyNywiIiwyLHsib2Zmc2V0Ijo0fV1d
\[\begin{tikzcd}
	{} & \dots & \dots & \dots & \dots & \dots & \dots \\
	& 0 & 0 & 0 & 0 & 0 & \dots \\
	& {R\{ dz\}} & 0 & {R\{xdz\}} & 0 & {R\{x^2dz\}} & \dots \\
	{} & R & 0 & {R\{x\}} & 0 & {R\{x^2\}} & \dots \\
	{} &&&&&& {}
	\arrow[from=4-4, to=3-2]
	\arrow[from=4-6, to=3-4]
	\arrow[shift left=4, from=5-1, to=5-7]
	\arrow[shift right=4, from=5-1, to=1-1]
\end{tikzcd}\]
\\
Since the $E^2$-page is already concentrated in the first two rows, there is only room for the $d^2$-differential (recall in Serre grading $d^r$ goes $r$ to the left and $r-1$ up.)
Furthermore because multiplicatively everything with potential for non-vanishing differential is generated by $x$ and the spectral sequence is multiplicative, it suffices to identify 
\[
d^2: R\{x\}\ra R\{dz\}. 
\]
If $R$ has equal characteristic, this differential actually has to vanish. The reason for this is that by the construction of the spectral sequence, we know that the edge homomorphism is given by $\THH_*(R;\Z_p)\ra \THH_*(R/\S[z];\Z_p)$ and everything that lies in the image of this map, has to be a permanent cycle. Because $R$ has equal characteristic, we can precompose the map by $\THH(k)\ra \THH(R;\Z_p)$ and choose $x\in \THH_2(R/\S[z];\Z_p)$ to be the image of the Bökstedt element under the composition of the two maps. Thus $x$ has to be a permanent cycle and $d^2$ vanishes. We hence get that $\THH_*(R;\Z_p)=R[x]\otimes_{\Z_p} \Lambda_{\Z_p}(dz)$, in particular $\THH_i(R;\Z_p)=R$ for all $i\geq 0$.
\\
For mixed characteristic $R$ we will identify it in the following Lemma.
Recall from Proposition \ref{description mixed char CDVR} that every mixed characteristic CDVR $R$ with perfect residue field $k$ has the form $R=W(k)[z]/E(z)$, where $E(z)\in W(k)[z]$ is an Eisenstein polynomial.
%\footnote{In this case $\pi$ is a uniformizer of $R$. For example $\Z_p[\sqrt[n]{p}]=W(\F_p)[\pi]$, $ E(z)=z^n-p.$}
\begin{lem} \label{differential}
For a mixed characteristic CDVR of the form $R=W(k)[z]/E(z)$, there is a generator $x\in \THH_2(R;\Z_p)$ with differential $d^2(x)=E'(\pi)dz$.
\end{lem}
\begin{proof}
We want to work relative to $\S_{W(k)}$ because this will enable us to make use of the description $R=W(k)[z]/E(z)$. To do this we again employ the base change for relative $\THH$:
\[
\THH(R/\S_{W(k)};\Z_p)\stackrel{\ref{relative THH formula}}{=}\THH(R;\Z_p)\otimes_{\THH(\S_{W(k)};\Z_p)} \S_{W(k)}
\]
This does not look helpful so far but we can simplify it further.
In the proof of Theorem \ref{Bökstedt for perfect rings} we already saw that $\F_p\otimes_\S \THH(\S_{W(k)};\Z_p)=k$. We also have a map $ \S_{W(k)}\ra \THH(\S_{W(k)};\Z_p) $. Both spectra are connective and $p$-complete, so we can check equivalence on $\F_p$-homology. There it is an isomorphism by the above computation. Thus we get
\[
\THH(R/\S_{W(k)};\Z_p)=\THH(R;\Z_p).
\]
As $R$ is $p$-complete and finite type over $\S_{W(k)}$, we can even drop the $p$-completion by part 4 of Proposition \ref{PropertiesTHH}. 
We can now identify the differential using that Hochschild homology and topological Hochschild homology agree in low degrees (Lemma \ref{PropertiesTHH}.3)):
\[
\THH_1(R;\Z_p)=\THH_1(R/\S_{W(k)})=\HH_1(R/W(k)).
\]
Furthermore Hochschild homology and Kähler differentials always agree in degree 1 and Kähler differentials are easy to compute, since $R=W(k)[z]/E(z)$.
\[
\HH_1(R/W(k))=\Omega_{R/W(k)}^1=
%\Omega_{(W(k)[z]/E(z))/W(k)}^1=
R\{dz\}/E'(\pi)dz.
\]
By the spectral sequence we also know 
\[
\THH_1(R;\Z_p)=R\{dz\}/\im(d^2:E_{2,0}^2\ra E_{0,1}^2),
\]
which finally implies that $\im(d^2:E_{2,0}^2\ra E_{0,1}^2)$ is equal to the submodule of $R\{dz\}$ generated by $E'(\pi)dz$.
This only pinpoints the generator of this submodule up to a unit, which we can choose.
\end{proof}
This result will easily allow us to compute $\THH_*(R;\Z_p)$, which we will do momentarily. But let us first issue the following warning.
\begin{warn}
The result of the Lemma implicitly depends on the choice of a uniformizer.
While this is not a problem if we are only interested in $\THH_*$ of one particular CDVR, it is a problem if we are dealing with a map $R\to S$ of two of those.
We can only determine the effect on $\THH_*$ of those maps, which preserve the chosen uniformizers. 
Only in this case is the map $R\to S$ actually a map of $\S[z]$-modules with respect to the module structures induced by the choices of the respective uniformizers.
Consider for example the inclusion $\Z_p\hookrightarrow \Z_p[\sqrt{p}]$. 
Any uniformizer in $\Z_p$ has the form $up$ for $u\in \Z_p^\times$, but of course none of them generate the maximal ideal of $\Z_p[\sqrt{p}]$. 
There is therefore no choice of uniformizers that is compatible with this map and our method is not able to calculate the induced map $\THH(\Z_p;\Z_p)\ra \THH(\Z_p[\sqrt{p}];\Z_p)$.
The same problem occurs for $\Z_p\hookrightarrow \Z_p[\sqrt[n]{p}],\ n\geq 2$. \\
But even if we can choose the uniformizers such that one maps onto the other as for example in the case $\Z_p[\sqrt{p}]\to \Z_p[\sqrt{p}], \ \sqrt{p}\mapsto -\sqrt{p}$, this is of not much use. For a meaningful calculation, we would need to choose the same uniformizer in domain and target. Compare this to the analogous situation for vector spaces. If we want to find a matrix representation of a vector space endomorphism, we should of course choose the same basis in domain and target. Otherwise every automorphism can be represented by the identity matrix when taking any basis in the domain and the image of this basis in the target.
\end{warn}
Let us now state the evaluation of the spectral sequence for mixed characteristic CDVRs. 
\begin{prop} \label{LM}
Let $R$ be a mixed characteristic complete discrete valuation ring with perfect residue field of characteristic $p$ and uniformizer $\pi$ with minimal polynomial $E$. The topological Hochschild homology groups are then given by
\[\THH_i(R;\Z_p)=
\begin{cases}
R &\text{for } n=0 \\
R/nE'(\pi)&\text{for } i=2n-1, n\geq 1 \\
0 & \text{else } 

\end{cases}
\]
\end{prop}
In the case $R=\Z_p$, we can make this more explicit using the Chinese remainder theorem\footnote{together with knowledge of the quotients: $\Z_p/n=0$ for $p\nmid n$ and $\Z_p/p^k=\Z/p^k$.}: $\THH_{2n-1}(\Z_p;\Z_p)=\Z_p/n\Z_p=\Z/p^{\nu_p(n)}$, where $\nu_p$ denotes the $p$-adic valuation. Emplyoing a fracture square, this can also be used to recover Bökstedt's calculation of $\THH_*(\Z)$.
\begin{proof}[Proof of Proposition \ref{LM}]
By Lemma \ref{differential} and the multiplicativity of the spectral sequence we know that $d^2(x^n)=nd^2(x)x^{n-1}=nE'(\pi)x^{n-1}dz$.
All higher differentials vanish for degree reasons, so the $E^3$-page is already the $E^\infty$-page, which takes the form
% https://q.uiver.app/?q=WzAsMjMsWzEsMywiUiJdLFsyLDMsIjAiXSxbMiwyLCIwIl0sWzMsMywiMCJdLFsxLDIsIlIvXFxQaGknKFxccGkpIl0sWzMsMiwiUi8yXFxQaGknKFxccGkpIl0sWzQsMiwiMCJdLFs0LDMsIjAiXSxbNSwzLCIwIl0sWzUsMiwiUi8zXFxQaGknKFxccGkpIl0sWzYsMywiXFxkb3RzIl0sWzYsMiwiXFxkb3RzIl0sWzAsNF0sWzAsM10sWzYsNF0sWzAsMF0sWzAsMV0sWzEsMSwiMCJdLFsyLDEsIjAiXSxbMywxLCIwIl0sWzQsMSwiMCJdLFs1LDEsIjAiXSxbNiwxLCJcXGRvdHMiXSxbMTIsMTQsIiIsMCx7Im9mZnNldCI6LTR9XSxbMTIsMTUsIiIsMix7Im9mZnNldCI6NCwic2hvcnRlbiI6eyJ0YXJnZXQiOjEwfX1dXQ==
\[\begin{tikzcd}
	{} \\
	{} & 0 & 0 & 0 & 0 & 0 & \dots \\
	& {R/E'(\pi)} & 0 & {R/2E'(\pi)} & 0 & {R/3E'(\pi)} & \dots \\
	{} & R & 0 & 0 & 0 & 0 & \dots \\
	{} &&&&&& {}
	\arrow[shift left=4, from=5-1, to=5-7]
	\arrow[shift right=4, shorten >=7pt, from=5-1, to=1-1]
\end{tikzcd}\]
There is no room for extensions and we can directly read of the result.
\end{proof}

\begin{rmk}
    The above result was first obtained in \cite[Theorem 5.1]{LMTHHofnumberrings} using an entirely different spectral sequence due to Brun. They describe it in terms of the \textbf{inverse different} 
    \begin{equation*}
        \THH_{2n-1}(R;\Z_p)=\cD_R^{-1}/n.
    \end{equation*}
    The inverse different $\cD_R^{-1}$ of $R$ (or more precisely, the inverse different relative to the Witt vectors of the residue field $\cD_{R/W(k)}^{-1}$) is a fractional ideal of $R$. As the name suggests it is the inverse fractional ideal of the different $\cD_R$, which is an ideal of $R$. The different can be defined as the annihilator ideal of the module of Kähler differentials (see \cite[Chapter~3.7]{SerreLocal})
    \begin{equation*}
        \cD_R\coloneqq\Ann_R(\Omega^1_{R/\Z_p})
    \end{equation*}
    We can rewrite the annihilator ideal to make its functoriality more clear
    \begin{equation*}
        \Ann_R(\Omega^1_{R/\Z_p})=\ker\left(R\to \End(\Omega^1_{R/\Z_p})\right).
    \end{equation*} Here we consider elements of $R$ as endomorphisms via their multiplication action. Since $M\mapsto \End(M)=\Hom(M,M)$ is a combination of a covariant and a contravariant functor, it is only functorial in automorphisms. 
    Thus the description $\THH_{2n-1}(R;\Z_p)=\cD_R^{-1}/n$ only has the chance to be natural for automorphisms $R\rai R$. But even in this case, both sides behave completely differently. In degree 1, we have $\THH_1(R;\Z_p)=\HH_1(R;\Z_p)=\Omega^1_{R/\Z_p}$. Thus for an automorphism $\varphi:R\to R$ the induced map on Kähler differentials is given by the derivative $d\varphi$, while the map $\cD_R\to\cD_R$ is simply the restriction of $\varphi$ to $\cD_R$.
    For a concrete counterexample, consider the automorphism $\varphi: \Z_p[\zeta_p]\to \Z_p[\zeta_p],\ \zeta_p\mapsto \zeta_p^2 \quad (\text{for } p\neq 2)$. This shows that the isomorphism $\THH_1(R;\Z_p)\cong \cD_R^{-1}/R$ cannot be a natural isomorphism. \\
    Nevertheless the description using the inverse different, while not functorial, is still useful because it allows us to deduce statements for $\THH$ of rings of integers in number fields. For these rings, the definition of the different as the annihilator ideal of the Kähler differentials still makes sense if we take the Kähler differentials relative\footnote{This should then more appropriately be called the diffent relative to $\Z$, while the different we discussed before is the different relative to $\Z_p$ or the Witt vectors of the residue field.}
    to $\Z$ instead of $\Z_p$. It is then possible to express this different ideal in terms of the different ideals of all the completions of the ring of integers. This allowed Lindenstrauss and Madsen to give the following expression: For $A$ the ring of integers in a number field, we have
    \begin{equation*}
        \THH_{2n-1}(A)=\cD_A^{-1}/nA.
    \end{equation*}
\end{rmk}

For the case of quotients $R'=R/\pi^n$, we can initially proceed similarly: Construction \ref{descentSS} gives us a spectral sequence which by Proposition \ref{relative Bökstedt for quotients} takes the form 
\begin{equation*}
    E^2=R'[x]\langle y\rangle \otimes_{\Z_p} \Lambda_{\Z_p}(dz)\Rightarrow \THH_*(R').
\end{equation*}
    
We can again identify the $d^2$ differential, which is the only one that can occur
\[d^2(x)=E'(\pi)dz, \ d^2(y)=k\pi^{k-1}dz.
\]
Thus we get an explicit differential graded algebra whose homology computes $\THH_*(R')$. But depending on $n$ there is no longer a closed-form expression for its homology. For explicit examples see \cite[Chapter~6]{KNBokstedt}.
