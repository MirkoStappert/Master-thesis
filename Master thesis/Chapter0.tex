\chapter*{Introduction}
K-theory is a fundamental invariant with many deep connections to number theory, algebraic topology and algebraic geometry. It is also very hard to compute. Away from the characteristic motivic methods provide powerful tools to compute K-theory via the motivic spectral sequence and the identification of motivic and étale cohomology in high degrees due to the Lichtenbaum-Quillen conjecture (now a theorem proved by Rost-Veovodsky). At the characteristic trace methods have proven to be the most effective tool of attack. 
The cyclotomic trace map $K(R)\ra \TC(R)$ from K-theory to topological cyclic homology often provides a good approximation (\cite{HMWV}) and has in some cases lead to a complete identification of K-theory (see eg. \cite{HMlocalfields}). Thus the calculation of topological cyclic homology is also of fundamental interest. The first step consists of identifying the topological Hochschild homology. Besides the relation to K-theory, topological Hochschild homology is also of independent interest in p-adic Hodge theory (\cite{BMS2}). There are large classes of rings, for which it is identified. 
\\
Following chapter of my master thesis reviews the computation of Krause and Nikolaus in \cite{KN} 


