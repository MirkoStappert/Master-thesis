\chapter{Bousfield localization and $p$-completion of spectra}
Sources: Lurie chromatic htpy theory, Lecture 20, Bousfield: "The localization of spectra wrt homology", nlab: Bousfield localization of spectra, May Ponto: Chapter 10, https://arxiv.org/pdf/1712.07633.pdf, Bauer arithmetic fracture squares, 
Barnes-Roitzheim Foundations of stable htpy theory: (8.4.1)
\bi
\item More general framework: Bousfield loc. in Model categories at a specified set(class?) of morphisms $\cS$ 
\item Unification of two known "zooming methods" in algebraic geometry/commutative algebra, namely completion and localization
\ei

\begin{defn}
Let $E$ be a spectrum.
Another spectrum $X$ is called \textbf{$E$-acyclic}, if $E\otimes X\simeq 0$, i.e. the $E$-homology of $X$ is zero. The spectrum $X$ is \textbf{$E$-local}, if every map $Y\rightarrow X $ out of an $E$-acyclic spectrum $Y$ is nullhomotopic. A map of spectra is an \text{$E$-equivalence}, if it is an isomorphism on $E$-(co)homology. An $E$-localization of $X$ consists of a map of spectra $X\rightarrow L_E X$, such that $L_E X$ is $E$-local and $X\rightarrow L_E X$ is an $E$-equivalence. 
\end{defn}
\begin{exa}
a) Let $E=M\Z_{(p)}$ the Moore spectrum of the integers localized away from a prime $p$. Then $E$-localization is called $p$-localization.
A spectrum $X$ is called $p$-local, if $X\ra L_EX$ is an equivalence, i.e. it is $M\Z_{p}$-local.\newline
b) Let $E=M\Q$ the Moore spectrum of the rationals. Then $E$-localization is called rationalization. \newline
c) Let $E=\S/p=M\F_p$. Then $E$-localization is called $p$-completion. A spectrum $X$ is called $p$-complete, if $X\ra L_EX$ is an equivalence, ie. it is $\S/p$-local.
(nonsmashing in general) (for connective: The same as localization at  $E=H\F_p$)
\newline
d) Another important example is localization at Morava K-theories $K(n)$ in chromatic homotopy theory.
\end{exa}
We have the following basic Lemma:
\begin{lem}
a) A spectrum $X$ is $E$-local if and only if for every $E$-equivalence $Y \ra Z$ the induced map $[Z,X]\ra [Y,X]$ is an isomorphism.
\newline
b) $X\ra L_E X$ is an $E$-localization iff $X\ra L_EX$ is an initial map from $X$ to an $E$-local spectrum iff $X\ra L_EX$ is a final map from $X$ that is an $E$-equivalence.\newline
c) If $E$ is a ring spectrum, then every $E$-module $M$ spectrum is $E$-local.
\end{lem} 
\begin{proof}
a) If $Y$ is $E$-acyclic, then $X\ra 0$ is an $E$-equivalence, so by assumption $0=[0,X]\rai [Y,X]$.
\newline
Conversely, the fiber of an $E$-equivalence $\fib:=\fib(Y\ra Z)$ is $E$-acyclic. Applying $[-,X]$ to the fiber sequence $\fib\ra Y\ra Z $
gives us a short exact sequence of graded abelian groups: $0\ra [Z,X]\ra [Y,X] \ra [\fib, X]\ra 0$. Now it is clear, that $[\fib,X]=0$ is equivalent to $[Z,X]\ra [Y,X]$ being an isomorphism.
\newline
b) Fiber argument.
\newline
c) Given a map $Y\xrightarrow{f} M$ from an $E$-acyclic spectrum $Y$. We can factor it as $X\simeq\S\otimes X\xrightarrow{\eta\otimes 1}E\otimes X\xrightarrow{1\otimes f}E\otimes M\xrightarrow{\nu}M$, where $\eta, \nu$ are respectively the unit and multiplication map. 
By assumption $E\otimes X\simeq0$, so the composite is nullhomotopic.
\end{proof}

\begin{lem}
The class of $E$-local objects is closed under retracts and limits.
\end{lem}
%But not in general under hocolims. This holds iff smashing

\begin{lem}
Prüfer group: \begin{equation*}
    p^\infty\text{-torsion subgroup of } U(1) =:C_{p^\infty}=\Q_p/\Z_p=\Q/\Z_{(p)}=\Z[1/p]/\Z
\end{equation*}
\end{lem}
\begin{defn}
A group $A$ is called derived $p$-complete, iff $\Hom(C_{p^\infty},A)=0$ and the connecting map (from $0\ra \Z\ra \Z[1/p]\ra C_{p^\infty}\ra 0$) is an isomorphism $\Ext(C_{p^\infty},A)\lai A$. 
(Derived functors, for bounded p-power torsion is equivalent to classical p-complete, but in general weaker notion, forms abelian category).
\newline
Derived p completion of $A$ is $\Ext(C_{p^\infty},A)$??
\end{defn}
\begin{lem}
Equivalent definitions of derived p-complete, see Felix masterarbeit. May, Ponto Kapitel 10.
\end{lem}
\begin{thm}
A spectrum is $p$-complete, iff its homotopy groups are derived $p$-complete. 
\end{thm}
