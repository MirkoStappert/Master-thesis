\chapter{Functoriality for THH of CDVRs}
In this chapter, we will investigate functorial computations of (topological) Hochschild homology. \\
In the first section we will give a fully functorial description of the Hochschild homology groups. In the second section we will employ the Tor-spectral sequence and use its functoriality to give a description of the functoriality for $\THH$. 
%We can relate a certain part of this spectral sequence with a similar spectral sequence for $\HH$. 
\\ In the third and last section we will discuss a 'partial' functoriality of the descent spectral sequence from section 3.4, which allows us to compute $\THH$ for certain maps.
%We will then compare these computations to the ones of section two.
% %%%%%%%%%%%%%%%%%%%%%%%%%%%%%%%%%%%%%%%%%%
% \section{Functoriality of $\HH$ for CDVRs}
% %%%%%%%%%%%%%%%%%%%%%%%%%%%%%%%%%%%%%%%%%%
% \label{HHfunctoriality}
% In this section we will describe the functoriality of Hochschild homology. To do this, we will make use of the HKR-filtration, which is expressed in terms of \textbf{nonabelian derived functors}. They are a generalization of ordinary derived functors, which have the advantage that they also work in non-additive settings, i.e. for non-additive categories or non-additive functors. This originally goes back to \cite{DoldPuppeHomologie}, for a modern formulation see \cite[Section~5.5.8]{HigherToposTheory} or \cite{CSpurityanimation}, where the terminology \textit{animation} is used.
% The main idea is to use simplicial resolutions, which make sense in every category, instead of resolutions by chain complexes, which are only sensible in an additive setting\footnote{In the case of non-additive functors between additive categories, we can obtain a simplicial resolution by applying the Dold-Kan correspondence to a projective resolution.}. \\
% Two functors that we want to derive are the non-additive functor $\Lambda_R^n:\Mod_R\to \Mod_R$ (here we really mean an ordinary ring and the 1-category of ordinary modules over it) and Kähler differentials $\Omega^1_{-}:\CRing\to \Ab$ (note that $\Omega^1_R$ has a $R$-module structure). Deriving these provides ($\infty$-) functors $L\Lambda_R^n:D(R)_{\geq 0}\to D(R)_{\geq 0}$ and $\bL_{-}:s\CRing\to D(\Z)_{\geq 0}$. 


% % We will only need the concept to derived the non-additive exterior product functor $\Lambda^n_R:\Mod_R\to \Mod_R$, for $R$ an ordinary ring and $\Mod_R$ the 1-category of $R$-modules. The nonabelian derived functor is then a functor $L\Lambda^n_R:D(R)_{\geq 0}\to D(R)_{\geq 0}$. 
% % The general recipe to compute it goes as follows\footnote{In general we would need to resolve by finitely generated, projective modules, but since $\Lambda_R^n$ preserves filtered colimits this is not necessary in out case.}: Take a complex (e.g. an ordinary module concentrated in degree 0) find a resolution by a chain complex of projective $R$-modules, use the Dold-Kan correspondence (see e.g. \cite[Section~1.2.3]{lurie2017higher}) to turn this complex into an simplicial $R$-module. 
% % Then apply the functor $\Lambda^n_R$ levelwise to this simplicial object and finally convert it back into a chain complex using the inverse of the Dold-Kan correspondence. \\
% % The difference to ordinary derived functors only consists in the fact that we apply the functor that we want to derive \textit{after} turning the projective resolution into an simplicial object. 
% % As in the ordinary case the value does not depend on the choice of projective resolution.


% % \begin{enumerate}
% %     \item Write down pushout square, apply HH to it
% %     \item Need to compute derived TP, 
% %     \item Resolve $\Z_p$ over exterior algebra $\Lambda_{\Z_p}(e)$ gives divided power algebra
% %     \item TP then gives same DP algebra but with different differential
% %     \item Compute Homology of this, concentrated in odd degrees and given by the same group
% %     %\item But different functoriality, power.
% % \end{enumerate}

% \begin{lem}
%     For a CDVRs $R$ the cotangent complex relative to $\Z_p$ is concentrated in degree 0 and given by Kähler differentials
%     \begin{equation*}
%         \bL_{R/\Z_p}=\Omega^1_{R/\Z_p}[0].
%     \end{equation*}
% \end{lem}
% \begin{proof}
%     Using Proposition \ref{description mixed char CDVR}, we get a map $\Z_p[z]\to R\simeq \Z_p[z]/E$. 
%     The maps $\Z_p\to \Z_p[z]\to R$ then induce the fiber sequence
%     \begin{equation*}
%         \bL_{\Z_p[z]/\Z_p}\otimes_{\Z_p[z]}R\to \bL_{R/\Z_p}\to \bL_{R/\Z_p[z]}.
%     \end{equation*}
%     By general facts about the cotangent complex, we know that $\bL_{\Z_p[z]/\Z_p}=\Omega^1_{\Z_p[z]/\Z_p}[0]$ (this is a free $\Z_p[z]$-modules, so we do not need to derive the tensor product) and $\bL_{R/\Z_p[z]}=(E)/(E))^2[1]$.
%     Since they are concentrated in degree 0 and 1 respectively, $\bL_{R/\Z_p}$ can also only have homology in degrees 0 and 1. To prove the statement we thus have to check, that $H_1(\bL_{R/\Z_p})=0$. Let us look at the long exact sequence associated to the fiber sequence.
%     \begin{equation*}
%         0=H_1(\bL_{\Z_p[z]/\Z_p})\to H_1(\bL_{R/\Z_p})\to (E)/(E)^2 \xrightarrow{\delta} \Omega^1_{\Z_p[z]/\Z_p}\otimes_{\Z_p[z]}R.
%     \end{equation*} 
%     The map $\delta$ sends the class of $E$ to $E'dz\otimes 1$ and is thus injective, which implies that $H_1(\bL_{R/\Z_p})=0$.
% \end{proof}
% Section 2.1 of the thesis \cite{OnderiveddeRhamcohomoology} contains this example and many more as well as a recollection of the necessary tools to compute the cotangent complex.

% \begin{prop}
%     For a CDVR $R$ we have the natural\todo{Not true, need to modify} isomorphism 
%     \begin{equation*}
%         \HH_{2n-1}(R;\Z_p)=\HH_{2n-1}(R/\Z_p)=H_{n-1}(L\Lambda^{n}_R \Omega^1_{R/\Z_p}).
%     \end{equation*} 
% \end{prop}
% \begin{proof}
%     We will use the HKR filtration on $\HH(R/\Z_p)$, in particular the spectral sequence that this filtration induces. The associated graded of the HKR filtration is given by $\gr^i=L\Lambda^i_R\bL_{R/\Z_p}[i]$. By the previous Lemma, we have $\bL_{R/\Z_p}=\Omega^1_{R/\Z_p}$.
%     The associated spectral sequence thus takes the form
%     \begin{equation*}
%         E^1_{i,j}=\pi_i(L\Lambda^j_R(\Omega^1_{R/\Z_p}))\Rightarrow \HH_{i+j}(R/\Z_p)
%     \end{equation*}
%     Because we have a two term resolution of $\Omega^1_{R/\Z_p}$, the simplicial $R$-module $L\Lambda^j_R(\Omega^1_{R/\Z_p})$ has only degenerate simplices in dimension $j$ and higher. Therefore $H_i(L\Lambda^j_R(\Omega^1_{R/\Z_p}))=0$ for $j\geq i$
%     %\todo{or is it i-1?}
%     But also all the lower homology groups necessarily vanish, because $\Lambda_R^i(R^j)=0$ for $j<i$. \\
%     Hence we can see, that the $E^1$-page of the spectral sequence is (besides an $R$ in degree $(0,0)$) concentrated on the line $(n+1,n)$, where it is given by $H_{n-1}(L\Lambda^{n}_R \Omega^1_{R/\Z_p})$.
%     All differentials are necessarily 0 and the spectral sequence degenerates on the $E^1$-page with no room for extension problems.
% \end{proof}
% Using this and computing the nonabelian derived functor of the Kähler differentials.


% \begin{lem}
%     For the rings in consideration we have
%     \begin{equation*}
%         \HH_{2n-1}(R;\Z_p)=\Omega^1_{R/\Z_p}\simeq R/E'(\pi).
%     \end{equation*}
% \end{lem}
% \begin{proof}
%     By the previous lemma, we want to compute $H_{n-1}(L\Lambda^{n}_R \Omega^1_{R/\Z_p})$. A projective resolution of the $R$-module $\Omega^1_{R/\Z_p}$, is provided by 
%     \begin{equation*}
%        R\xrightarrow{\cdot E'(\pi)} R\to \Omega^1_{R/\Z_p}. 
%     \end{equation*}
%     Using the Dold-Kan correspondence, we get a simplicial $R$-module with $R^{n+1}$ in degree $n$. Applying $\Lambda_R^n$ levelwise and converting it back into gives a chain complex again. We want to compute $H_{n-1}$ of it, so we only need the degrees $n-2,n-1,n$, which look as follows 
%     \begin{equation*}
%         \Lambda_R^nR^{n-1}=0\la \Lambda_R^n R^n=R \xleftarrow{d} \Lambda_R^n R^{n+1}=R^{n+1}. 
%     \end{equation*} 
%     Chasing through the Dold-Kan construction we see that the differential is given by multiplication with $E'(\pi)$. Thus $H_{n-1}\left( L\Lambda_R^n(\Omega^1_{R/\Z_p}) \right)=R/E'(\pi)$, which by the last lemma is exactly $\HH_{2n-1}(R;\Z_p)$.
% \end{proof}
% Furthermore, we obtain the following Proposition, which gives a fully functorial description of $\HH$ on the category of CDVRs with chosen uniformizer and all maps (not necessarily preserving the choice of uniformizer). For a map $\varphi: R\to S$, we can express this in terms of $\varphi$-semilinear\footnote{Recall: Let $R$, $S$ be rings, $M$ a module over $R$, $N$ a module over $S$ and $\varphi:R\to S$ a map of rings. Then an additive map $f:M\to N$ is called $\varphi$-semilinear if $f(rm)=\phi(r)f(m)$ for all $m\in M,\ r\in R$} maps.

%%%%%%%%%%%%%%%%%%%%%%%%%%%%%%%%%%%%%%%%%%
\section{Functoriality of $\HH$ for CDVRs}
%%%%%%%%%%%%%%%%%%%%%%%%%%%%%%%%%%%%%%%%%%
\label{HHFunctoriality}
In this section we will deal with the functoriality of Hochschild homology for CDVRs.
In the same vein as Chapter 3, the computation of absolute Hochschild homology will first pass through an auxiliary relative result. While archieving this, we also need to be attentive to the functoriality. So what is the correct domain category for $\HH(-;\Z_p[z])$?
The first answer one might come up with is probably the category of $\Z_p[z]$-algebras. But as we have previously seen, this is not suitable for our purposes, because the morphism sets are too restrictive. Given two CDVRs, we want to talk about arbitrary maps between them, but only maps that preserve choices of uniformizers (which give the $\Z_p[z]$-algebra structures) are $\Z_p[z]$-algebra homomorphisms. We will therefore now introduce a more refined category, which allows us to also consider these maps. \\
For ease of notation, we first only deal with CDVRs with residue field $\F_p$, i.e. purely ramified extensions of $\Z_p$. We discuss the general case in Remark \ref{generalresiduefield}
\\
Let $\CDVRL$ be the category, whose \textit{objects} are 
given by pairs $(R,\pi)$, where $R$ is a complete discrete valuation rings of mixed characteristic with residue field $\F_p$ and $\pi$ is a uniformizer of $R$. The element $\pi$ determines a surjective map $ev_\pi:\Z_p[z]\to R, \ z\mapsto \pi$. A \textit{morphism} between to objects $(R,\pi)$ and $(S,\epsilon)$ is now given by a pair $(\phi,f)$, where $\phi:R\to S$ and $f:\Z_p[z]\to \Z_p[w]$ are $\Z_p$-algebra homomorphisms, that make the following diagram commutes.
% https://q.uiver.app/?q=WzAsNCxbMCwxLCJSIl0sWzEsMCwiXFxaX3Bbd10iXSxbMCwwLCJcXFpfcFt6XSJdLFsxLDEsIlMiXSxbMiwwLCJldl9cXHBpIl0sWzEsMywiZXZfXFxlcHNpbG9uIl0sWzAsMywiXFxwaGkiXSxbMiwxLCJmIl1d
\[\begin{tikzcd}
	{\Z_p[z]} & {\Z_p[w]} \\
	R & S
	\arrow["{ev_\pi}", from=1-1, to=2-1]
	\arrow["{ev_\epsilon}", from=1-2, to=2-2]
	\arrow["\phi", from=2-1, to=2-2]
	\arrow["f", from=1-1, to=1-2]
\end{tikzcd}\]
The commutativity expresses, that $f$ is a \textit{lift} of $\phi$, hence the name for the category.
Giving the map $f:\Z_p[z]\to \Z_p[w]$ is equivalent to providing a polynomial in $\Z_p[w]$, which we abusively also call $f$. The map is then given by $g(z)\mapsto g(f(w))$, i.e. by substitution. \\
Abstractly, this category is the full subcategory of the category of $\Z_p$-algebras on objects of the form $\Z_p[z]\to R$, where $R$ is a mixed characteristic CDVR with residue field $\F_p$ and the map sends $z$ to a uniformizer of $R$. 
Several constructions/computations, that we will now make will only be functorial in this input category. In particular $\HH_*(-;\Z_p[z])$ canonically provides a functor from $\CDVRL$ to graded $\Z_p$-algebras.
We also have the category $\CDVRu$, with the same objects as $\CDVRL$. Morphisms between objects $(R,\pi)$ and $(S,\epsilon)$ are given by a $\Z_p$-algebra homomorphisms $R\to S$ without a choice of lift and without any compatibility of the uniformizers, in particular it need not send one uniformizer on the other. There is an evident forgetful functor $U:\CDVRL\to \CDVRu$, which is the identity on objects and forgets the lift on morphisms, i.e. it sends $(\phi,f)$ to just $\phi$. ($p$-completed) Hochschild homology can be considered as a functor $\HH_*(-;\Z_p):\CDVRu\to \gr \Z_p$-$\Mod$ with values in the category of graded $\Z_p$-modules. Via precomposition with $U$ we obtain a functor with domain $\CDVRL$. Our main result is a natural computation of $U\circ \HH_*(-;\Z_p)$, i.e. we give a natural isomorphism from $\HH_*(R;\Z_p)$ to the homology of a very easy dga. 
This natural isomorphism is an equivalence of functors $\CDVRL\to \gr \Z_p$-$\Alg$. Since Hochschild homology is independent of the choice of lift, the homology of the dga is necessarily also indepent and hence descends to a functor $\CDVRu\to \gr \Z_p$-$\Alg$.
The upshot is, that our dga description can be used to functorially compute ($p$-completed) Hochschild homology by choosing a lift, but the such computed map does not depend of the choice of lift.
\\
The proof of the main result consists of the following steps:
\begin{itemize}
    \item Compute the second relative Hochschild homology $\HH_2(R/\Z[z];\Z_p)$.
    \item Establish that the full relative Hochschild homology is given by the divided power of the degree 2 part: $\HH_*(R/\Z[z];\Z_p)=\Gamma_R(\HH_2(R/\Z[z];\Z_p))$.
    \item Use a descent spectral sequence as in Section 3.4 to obtain the absolute Hochschild homology from the relative term. 
\end{itemize}
For the first two steps in the program, we will use the HKR-filtration, which is expressed in terms of \textbf{nonabelian derived functors}. They are a generalization of ordinary derived functors, which have the advantage that they also work in non-additive settings, i.e. for non-additive categories or non-additive functors. This originally goes back to \cite{DoldPuppeHomologie}, for a modern formulation see \cite[Section~5.5.8]{HigherToposTheory} or \cite[Section~5.1.4]{CSpurityanimation}, where the terminology \textit{animation} is used.
The main idea is to use simplicial resolutions, which make sense in every category, instead of resolutions by chain complexes, which are only sensible in an additive setting\footnote{In the case of non-additive functors between additive categories, we can obtain a simplicial resolution by applying the Dold-Kan correspondence to a projective resolution.}. \\
Two functors that we want to derive are the non-additive functor $\Lambda_R^n:\Mod_R\to \Mod_R$ (here we really mean an ordinary ring and the 1-category of ordinary modules over it) and Kähler differentials $\Omega^1_{-}:\CRing\to \Ab$ (note that $\Omega^1_R$ has a $R$-module structure). Deriving these provides ($\infty$-) functors $L\Lambda_R^n:D(R)_{\geq 0}\to D(R)_{\geq 0}$ and $\bL_{-}:s\CRing\to D(\Z)_{\geq 0}$. \\
Now recall that the HKR-filtration is a natural, multiplicative filtration on relative Hochschild homology $\HH(R/S)$, with associated graded given by $\gr^n_{\rm HKR}=( L\Lambda_R^n\bL_{R/S})[n]$ (see eg. \cite[Theorem~IV.4.1]{NS}).
\begin{prop} \label{FunctorialrelativeHH}
    Let $(R,\pi)\in \CDVRL$ and $I=\ker(\Z_p[z]\to R)$ the kernel of the $\pi$-substitution morphism. Then we have a natural isomorphism of functors from $\CDVRL$ to $\gr \Z_p$-$\Alg$ 
    \begin{equation*}
        \HH_*(R/\Z[z];\Z_p)=\HH_*(R/\Z_p[z])=\Gamma_R(I/I^2).
    \end{equation*}
\end{prop}
Let us prepare this result, by first establishing the computation in the second degree.
\begin{lem}
    Let $(R,\pi)\in \CDVRL$ and $I=\ker(\Z_p[z]\to R)$. For the second Hochschild homology, we have the following natural identification of functors from $\CDVRL$ to $\Z_p$-$\Mod$.
    \begin{equation*}
        \HH_2(R/\Z_p[z])=I/I^2
    \end{equation*}
\end{lem}
\begin{proof}
    Because of the presentation $R=\Z_p[z]/I$, we naturally have $\bL_{R/\Z_p[z]}=I/I^2[1]$ (see e.g. \cite[\href{https://stacks.math.columbia.edu/tag/08SJ}{Tag 08SJ}]{stacks-project}). Furthermore the HKR-filtration provides us with a natural isomorphism $\HH_2(R/\Z_p[z])=\bL_{R/\Z_p[z]}$, thus yielding the result. 
\end{proof}
The module $I/I^2$ is isomorphic to $R$. To describe the isomorphism, let $E\in \Z_p[z]$ be the minimal polynomial of $\pi$. Then $E$ generates the ideal $I=(E)$, so we also have $\HH_2(R/\Z_p[z])=(E)/(E^2)$. The isomorphism is now given by $R\rai I/I^2, 1\mapsto [E]$, with inverse $I/I^2\rai R, [g]\mapsto (g/E)(\pi)$. \\
The reason, that we write $I/I^2$ instead of $R$, is to indicate the correct functoriality.
Let us be completely explicit about this: Assume $(\phi,f):(R,\pi)\to (S,\epsilon)$ is a morphism in $\CDVRL$. Let $E\in \Z_p[z]$ and $F\in \Z_p[w]$ be minimal polynomials of $\pi$ and $\epsilon$ respectively. Under the identifications $R\simeq (E)/(E^2)$ and $S\simeq (F)/(F)^2$, the induced map on the relative second Hochschild homology 
\begin{equation*}
    R\rai (E)/(E)^2 \simeq\HH_2(R/\Z_p[z]) \to \HH_2(S/\Z_p[w]) \simeq (F)/(F)^2 \rai S
\end{equation*}
is given by sending $1\mapsto [E(z)]\mapsto [E(f(w))]\mapsto \frac{E(f(w))}{F(w)}|_{w=\epsilon}$. This uniquely determines the map by requiring, that it is $\phi$-semilinear\footnote{Recall: Let $R$, $S$ be rings, $M$ a module over $R$, $N$ a module over $S$ and $\varphi:R\to S$ a map of rings. Then an additive map $f:M\to N$ is called $\varphi$-semilinear if $f(rm)=\phi(r)f(m)$ for all $m\in M,$ and $\ r\in R$.}, i.e. we send $R\ni r\to \phi(r)\frac{E(f(w))}{F(w)}|_{w=\epsilon}$.


\begin{proof}[Proof of Proposition \ref{FunctorialrelativeHH}]
    The first isomorphism follows from Lemma \ref{PropertiesTHH}.5) since $R$ is $p$-complete and of finite type over $\Z_p[z]$.  \\
    For the secon isomorphism:
    By \cite[§5.4.3]{illusie2006complexe}, we have a natural isomorphism $L\Lambda_R^n(I/I^2[1])=\Gamma_R^n(I/I^2)[n]$. Thus the spectral sequence induced from the HKR-filtration looks as follows
    % https://q.uiver.app/?q=WzAsMzIsWzEsNSwiUiJdLFsyLDUsIjAiXSxbMiw0LCJJL0leMiJdLFszLDUsIjAiXSxbMSw0LCIwIl0sWzMsNCwiMCJdLFs0LDQsIjAiXSxbNCw1LCIwIl0sWzUsNCwiXFxkb3RzIl0sWzAsNl0sWzAsNV0sWzYsNl0sWzAsMl0sWzAsM10sWzEsMywiMCJdLFsyLDMsIjAiXSxbMywzLCJcXEdhbW1hX1IoSS9JXjIpIl0sWzQsMywiMCJdLFswLDFdLFsxLDIsIjAiXSxbMiwyLCIwIl0sWzMsMiwiMCJdLFs0LDIsIlxcR2FtbWFeMl9SKEkvSV4yKSJdLFs1LDEsIlxcZG90cyJdLFsyLDEsIlxcZG90cyJdLFswLDBdLFs1LDMsIlxcZG90cyJdLFs1LDIsIlxcZG90cyJdLFs1LDUsIlxcZG90cyJdLFszLDEsIlxcZG90cyJdLFs0LDEsIlxcZG90cyJdLFsxLDEsIlxcZG90cyJdLFs5LDExLCIiLDAseyJvZmZzZXQiOi00fV0sWzksMjUsIiIsMix7Im9mZnNldCI6NCwic2hvcnRlbiI6eyJ0YXJnZXQiOjEwfX1dXQ==
\[\begin{tikzcd}
	{} \\
	{} & \dots & \dots & \dots & \dots & \dots \\
	{} & 0 & 0 & 0 & {\Gamma^2_R(I/I^2)} & \dots \\
	{} & 0 & 0 & {\Gamma_R(I/I^2)} & 0 & \dots \\
	& 0 & {I/I^2} & 0 & 0 & \dots \\
	{} & R & 0 & 0 & 0 & \dots \\
	{} &&&&&& {}
	\arrow[shift left=4, from=7-1, to=7-7]
	\arrow[shift right=4, shorten >=12pt, from=7-1, to=1-1]
\end{tikzcd}\]
    There is no room for differentials and it degenerates on the $E^1$-page. We also do not have extension problems. The isomorphisms $L\Lambda_R^n(I/I^2[1])=\Gamma_R^n(I/I^2)[n]$ for varying $n$ are multiplicatively compatible, i.e. we have 
    $L\Lambda_R^*(I/I^2[1])=\Gamma_R(I/I^2)$. This determines the multiplicative structure on the $E^1=E^\infty$-page and hence gives the result.
\end{proof}
Let us again unravel, what this means concretely. The $R$-module $I/I^2$ is free of rank 1 with generator provided by the class of $E$, the minimal polynomial of the uniformizer. 
Let us instead denote $x\coloneqq [E]\in (E)/(E)^2$. We can then identify Hochschild homology with the free divided power algebra on a generator in degree two $\HH_*(R/\Z_p[z])\simeq \Gamma_R\{x\}$. Let now $(\phi,f):(R,\pi)\to (S,\epsilon)$ be a morphism in $\CDVRL$ and $E\in\Z_p[z],F\in\Z_p[w]$ minimal polynomials of $\pi,\epsilon$. The induced map on relative Hochschild homology 
\begin{align*}
    \Gamma_R\{x\}\simeq \HH_*(R/\Z_p[z])\to \HH_*(S/\Z_p[w])\simeq \Gamma_S\{y\}
\end{align*} 
is then uniquely determined by sending $x^{[n]}$, the $n$-the divided power of $x$, to $\left(\frac{E(f(w))}{F(w)}|_{w=\epsilon} \right)^n y^{[n]}$ and being $\phi$-semilinear.
\\
\indent Now we want to obtain the absolute Hochschild homology from the relative term. For this we proceed in the same way as in Section \ref{AbsoluteTHHofCDVRs}. 
Again, we have a descent spectral sequence for obtaining $\HH(R;\Z_p)$ out of $\HH(R/\Z_p[z])$.
This works out as follows: 
We use the trivial isomorphism $\HH(R;\Z_p)=\HH(R;\Z_p)\otimes_{\HH(\Z_p[z]/\Z_p)}\HH(\Z_p[z]/\Z_p)$. Then we filter $\HH(\Z_p[z]/\Z_p)$ by its Whitehead tower $\tau_{\geq \bullet}\HH(\Z_p[z]/\Z_p)$ and tensor this up to obtain the filtration $\HH(R;\Z_p)\otimes_{\HH(\Z_p[z]/\Z_p)} \tau_{\geq \bullet}\HH(\Z_p[z]/\Z_p)$ on $\HH(R;\Z_p)$.
The associated graded of this, is given by:
\begin{align*}
    \gr^j &=\HH(R;\Z_p)\otimes_{\HH(\Z_p[z]/\Z_p)} \Omega^j_{\Z_p[z]/\Z_p}\\
    &=
    \left( \HH(R;\Z_p)\otimes_{\HH(\Z_p[z]/\Z_p)} \Z_p[z] \right)\otimes_{\Z_p[z]}\Omega^j_{\Z_p[z]/\Z_p} \\
    &\stackrel{\ref{relative THH formula}}{=}
    \HH(R/\Z_p[z])\otimes_{\Z_p[z]}\Omega^j_{\Z_p[z]/\Z_p}
\end{align*}
The descent spectral sequence associated to this filtration thus takes the form\footnote{We start at the $E^2$-page, because the $d^1$-differential is trivial for degree reasons.}
\begin{equation*}
    E^2=\HH_i(R/\Z_p[z])\otimes_{\Z_p[z]}\Omega^j_{\Z_p[z]/\Z_p}\Rightarrow \HH_{i+j}(R;\Z_p)
\end{equation*}
% This is constructed in the same way as the other spectral sequence:
% We have $\HH(R)=\HH(R)\otimes_{\HH(\Z_[z])}\HH(\Z[z])$ and then filter $\HH(\Z[z])$ by its Whitehead tower. Associated graded given by Kähler differentials. Then spectral sequence of filtered chain complex.
Due to degree reasons, the spectral sequence degenerates at the $E^3$-page. Thus we can identify the absolute ($p$-completed) Hochschild homology with the homology of the dga given by the $E^2$-page of the spectral sequence equipped with the $d^2$-differential. I.e. we get a natural isomorphism of functors $\CDVRL\to \gr \Z_p$-$\Alg$
\begin{equation*}
    \HH_*(R/\Z_p)=H_*(\Gamma_R(I/I^2)\otimes_{\Z_p[z]}\Omega^*_{\Z_p[z]/\Z_p},\partial=d^2)
\end{equation*}
Because the right hand side is even a functor on $\CDVRu$, the left hand side also descends to one. Since the forgetful functor $\CDVRL\to \CDVRu$ is full\footnote{This holds because, $\Z_p[z]\to R$ is surjective, so we can lift every map $\phi:R\to S$ to a map $\Z_p[z]\to \Z_p[w]$} this allows us to compute the effect of Hochschild homology by choosing an arbitrary lift and then compute the functoriality via the homology of the dga. \\
The final thing, that remains to be done, is identifying the homology of the dga.
\begin{thm}
    Let $(R,\pi)\in \CDVRL$, $E\in \Z_p[z]$ a minimal polynomial of $\pi$ and let $x\coloneqq [E]\in (E)/(E)^2=\HH_2(R;\Z_p[z])$.
    %Under the identifications $HH_2n(R/\Z_p[z]\simeq R$ and 
    The differential of the descent spectral sequence $d^2:R\{x^{[n+1]}\}\to R\{x^{[n]}dz\}$ is injective and has image given by the principal ideal generated by $E'(\pi)x^{[n-1]}dz$. The nonzero Hochschild homology groups are therefore $\HH_0(R;\Z_p)=R$ and $\HH_{2n+1}\simeq R/(E'(\pi))$\\
    %\begin{equation*}
    %    d^2(x^{[n]})=E'(\pi)x^{[n-1]}dz
    %\end{equation*}
    Regarding the functoriality: Given another $(S,\epsilon)\in \CDVRL$, a map $(\phi,f):(R,\pi)\to (S,\epsilon) $,  $F\in \Z_p[w]$ the minimal polynomial of $\epsilon$ and denote $y\coloneqq [F]\in \HH_2(S;\Z_p[w])$. The functoriality is the completely encoded in the commutative diagram
    % https://q.uiver.app/?q=WzAsMTAsWzAsMCwiMCJdLFsxLDAsIlJcXHt4XntbbisxXX1cXH0iXSxbMiwwLCJSXFx7eF57W25dfWR6XFx9Il0sWzMsMCwiXFxUSEhfezJuKzF9KFI7XFxaX3ApIl0sWzQsMCwiMCJdLFswLDEsIjAiXSxbMSwxLCJTXFx7eV57W24rMV19XFx9Il0sWzIsMSwiU1xce3lee1tuXX1kd1xcfSJdLFszLDEsIlxcSEhfezJuKzF9KFM7XFxaX3ApIl0sWzQsMSwiMCJdLFswLDFdLFsxLDIsImReMiJdLFsyLDMsIiIsMCx7InN0eWxlIjp7ImhlYWQiOnsibmFtZSI6ImVwaSJ9fX1dLFszLDRdLFs1LDZdLFs2LDcsImReMiJdLFs3LDgsIiIsMCx7InN0eWxlIjp7ImhlYWQiOnsibmFtZSI6ImVwaSJ9fX1dLFs4LDldLFsxLDYsIlxcdmFycGhpIl0sWzIsNywiXFx2YXJwaGlcXG90aW1lcyBkZiJdLFszLDgsIiIsMSx7InN0eWxlIjp7ImJvZHkiOnsibmFtZSI6ImRhc2hlZCJ9fX1dXQ==
\[\begin{tikzcd}
	0 & {R\{x^{[n+1]}\}} & {R\{x^{[n]}dz\}} & {\HH_{2n+1}(R;\Z_p)} & 0 \\
	0 & {S\{y^{[n+1]}\}} & {S\{y^{[n]}dw\}} & {\HH_{2n+1}(S;\Z_p)} & 0
	\arrow[from=1-1, to=1-2]
	\arrow["{d^2}", from=1-2, to=1-3]
	\arrow[two heads, from=1-3, to=1-4]
	\arrow[from=1-4, to=1-5]
	\arrow[from=2-1, to=2-2]
	\arrow["{d^2}", from=2-2, to=2-3]
	\arrow[two heads, from=2-3, to=2-4]
	\arrow[from=2-4, to=2-5]
	\arrow[from=1-2, to=2-2]
	\arrow[from=1-3, to=2-3]
	\arrow[dashed, from=1-4, to=2-4]
\end{tikzcd}\]
The map $R\{x^{[n]}dz\}\to S\{y^{[n]}dw\}$ is given as the unique $\phi$-semilinear map, that acts on $ x^{[n]}dz$ as 
\begin{equation*}
    x^{[n]}dz\mapsto \left(\frac{E(f(w))}{F(w)}|_{w=\epsilon} \right)^n f'(\epsilon)y^{[n]}dw.
\end{equation*}

\end{thm}
\begin{proof}
    Let us start with a picture of the spectral sequence.
    % https://q.uiver.app/?q=WzAsMjgsWzEsMywiUiJdLFsyLDMsIjAiXSxbMiwyLCIwIl0sWzMsMywiUlxce3hcXH0iXSxbMSwyLCJSXFx7IGR6XFx9Il0sWzMsMiwiUlxce3hkelxcfSJdLFs0LDIsIjAiXSxbNCwzLCIwIl0sWzEsMSwiMCJdLFsyLDEsIjAiXSxbMywxLCIwIl0sWzQsMSwiMCJdLFs1LDMsIlJcXHt4XntbMl19XFx9Il0sWzUsMiwiUlxce3hee1syXX1kelxcfSJdLFs1LDEsIjAiXSxbNiwzLCJcXGRvdHMiXSxbNiwyLCJcXGRvdHMiXSxbNiwxLCJcXGRvdHMiXSxbMSwwLCJcXGRvdHMiXSxbMiwwLCJcXGRvdHMiXSxbMywwLCJcXGRvdHMiXSxbNCwwLCJcXGRvdHMiXSxbNSwwLCJcXGRvdHMiXSxbNiwwLCJcXGRvdHMiXSxbMCw0XSxbMCwzXSxbNiw0XSxbMCwwXSxbMyw0LCJkXjIiLDJdLFsxMiw1LCJkXjIiLDJdLFsyNCwyNiwiIiwwLHsib2Zmc2V0IjotNH1dLFsyNCwyNywiIiwyLHsib2Zmc2V0Ijo0fV1d
\[\begin{tikzcd}
	{} & \dots & \dots & \dots & \dots & \dots & \dots \\
	& 0 & 0 & 0 & 0 & 0 & \dots \\
	& {R\{ dz\}} & 0 & {R\{xdz\}} & 0 & {R\{x^{[2]}dz\}} & \dots \\
	{} & R & 0 & {R\{x\}} & 0 & {R\{x^{[2]}\}} & \dots \\
	{} &&&&&& {}
	\arrow["{d^2}"', from=4-4, to=3-2]
	\arrow["{d^2}"', from=4-6, to=3-4]
	\arrow[shift left=4, from=5-1, to=5-7]
	\arrow[shift right=4, from=5-1, to=1-1]
\end{tikzcd}\]
We see that it suffices to understand the effect of $d^2$ on $x,x^{[2]},x^{[3]},\dots$. Knowing that $\HH_1(R;\Z_p)\simeq R/E'(\pi)$, we obtain that $d^2$ must send $x$ to a generator of the ideal generated by $E'(\pi)$, i.e. $d^2(x)=uE'(\pi)$ for some $u\in R^{\times}$. By the Leibniz rule, we then also get $d^2(x^n)= n d^2(x)x^{n-1}$. But we need the differential on divided powers of $x$ and not of the actual powers. We claim, that $d^2$ is a PD-derivation, i.e. $d^2(x^{[n]})=x^{[n-1]}d^2(x)$. In fact, this follows from multiplicativity:
\begin{align*}
    0&=d^2(x^{[n]})-x^{[n-1]}d^2(x) \\
    \Longleftrightarrow 0&=n!(d^2(x^{[n]})-x^{[n-1]}d^2(x))\\
    &=d^2(n!x^{[n]})-n((n-1)!x^{[n-1]})d^2(x)\\
    &=d^2(x^n)-nx^{n-1}d^2(x)
\end{align*}
The equivalence follows, because all involved modules are torsion free. This proves the first part of the theorem.\\
For the functoriality: We get the map of short exact sequences from the functoriality of the descent spectral sequence. The map $R\{x^{[n]}dz\}\to S\{y^{[n]}dw\}$ is the tensor product of the map $\HH_{2n}(R;\Z_p[z])\to \HH_{2n}(S;\Z_p[w])$ and the map $\Omega^1_{\Z_p[z]}\xrightarrow{df}\Omega^1_{\Z_p[w]/\Z_p}$. We know the effect of the first map from Proposition \ref{FunctorialrelativeHH} and the map on Kähler differentials works as the formal derivative, i.e. $dz\mapsto d(f(w))=f'(w)dw$. 
The tensor product of these two maps is exactly the map, we claimed in the theorem.

\end{proof}
\begin{rmk} \label{generalresiduefield}
    Let us now indicate, which modification we need in the general case of an arbitrary perfect residue field of charactersitic $p$.
    We redefine $\CDVRL$ to be the full subcategory of the arrow category of $\Z_p$-algebras on objects of the form $W(k)[z]\to R$, where $R$ is a mixed characteristic complete discrete valuation ring with perfect residue field $k$, and the map sends $z$ to a uniformizer of $R$. Again, this data is equivalent to the choice of an $R$ and a uniformizer. Maps are 
\end{rmk}
\begin{rmk}
    In our discussion above, we actually nowhere used, that in the presentation $R=W(k)/E(z)$, the polynomial $E$ is Eisenstein. All results hold verbatim for arbitrary quotients of $W(k)[z]$ by principal ideals.
\end{rmk}
% \begin{prop}
%     \todo{This is wrong! Correct}
%     Let $R\xrightarrow{\varphi} S$ be a map of CDVRs. We get an induced map $\mathrm{d}\varphi:\Omega^1_{R/\Z_p}\to \Omega^1_{S/\Z_p}$. Choose uniformizers with respective Eisenstein polynomials $E,\tilde{E}$. These Eisenstein polynomials give a resolution of the Kähler differentials and by the fundamental theorem of homological algebra we can lift $\mathrm{d}\varphi$ to maps $f$ and $g$ on the terms of the resolutions
%      https://q.uiver.app/?q=WzAsNixbMSwwLCJSIl0sWzIsMCwiXFxPbWVnYV4xX3tSL1xcbWF0aGJie1p9X3B9Il0sWzAsMCwiUiJdLFswLDEsIlMiXSxbMSwxLCJTIl0sWzIsMSwiXFxPbWVnYV4xX3tTL1xcbWF0aGJie1p9X3B9Il0sWzAsMV0sWzIsMCwiXFxjZG90IEUoXFxwaSkiXSxbMyw0LCJcXGNkb3QgXFx0aWxkZXtGfShcXHRpbGRle1xccGl9KSJdLFs0LDVdLFsxLDUsIlxcbWF0aHJte2R9XFx2YXJwaGkiXSxbMCw0LCJmIiwwLHsic3R5bGUiOnsiYm9keSI6eyJuYW1lIjoiZGFzaGVkIn19fV0sWzIsMywiZyIsMCx7InN0eWxlIjp7ImJvZHkiOnsibmFtZSI6ImRhc2hlZCJ9fX1dXQ==
% \[\begin{tikzcd}
% 	R & R & {\Omega^1_{R/\mathbb{Z}_p}} \\
% 	S & S & {\Omega^1_{S/\mathbb{Z}_p}}
% 	\arrow[from=1-2, to=1-3]
% 	\arrow["{\cdot E(\pi)}", from=1-1, to=1-2]
% 	\arrow["{\cdot \tilde{E}(\tilde{\pi})}", from=2-1, to=2-2]
% 	\arrow[from=2-2, to=2-3]
% 	\arrow["{\mathrm{d}\varphi}", from=1-3, to=2-3]
% 	\arrow["f", dashed, from=1-2, to=2-2]
% 	\arrow["g", dashed, from=1-1, to=2-1]
% \end{tikzcd}\]
%     The map on Hochschild homology groups $\HH_{2n-1}(R;\Z_p)\to         \HH_{2n-1}(S;\Z_p)$ is now induced by the map $fg^{n-1}:R\to S,\ a\mapsto f(a)g(a)^{n-1}$.
% \end{prop}
% \begin{proof}
%     This is a matter of understanding the functoriality of the statement in the last lemma. We get the diagram
%      https://q.uiver.app/?q=WzAsNixbMSwwLCJcXExhbWJkYV9SXm5SXm4iXSxbMiwwLCIwIl0sWzEsMSwiXFxMYW1iZGFfU15uU15uIl0sWzIsMSwiMCJdLFswLDEsIlxcTGFtYmRhX1NeblNee24rMX0iXSxbMCwwLCJcXExhbWJkYV9SXm5SXntuKzF9Il0sWzAsMV0sWzIsM10sWzQsMiwiZCJdLFs1LDAsImQiXSxbNSw0XSxbMCwyXV0=
%     \[\begin{tikzcd}
% 	{\Lambda_R^nR^{n+1}} & {\Lambda_R^nR^n} & 0 \\
% 	{\Lambda_S^nS^{n+1}} & {\Lambda_S^nS^n} & 0
% 	\arrow[from=1-2, to=1-3]
% 	\arrow[from=2-2, to=2-3]
% 	\arrow["d", from=2-1, to=2-2]
% 	\arrow["d", from=1-1, to=1-2]
% 	\arrow[from=1-1, to=2-1]
% 	\arrow[from=1-2, to=2-2]
%     \end{tikzcd}\]
%     The relevant map $\Lambda_R^nR^n\to \Lambda_S^nS^n$ is induced by the map $R^n\to S^n$ given by the diagonal matrix $diag(g,\dots, g, f)$. This matrix has determinant $fg^{n-1}$, which is thus the induced map on homology.
% \end{proof}
%%%%%%%%%%%%%%%%%%%%%%%%%%%%%%%%%%%%%%%%%%%%%%%%%%%%%%%%%%%%%
\section{Functoriality of $\THH$ via the Tor spectral sequence}
%%%%%%%%%%%%%%%%%%%%%%%%%%%%%%%%%%%%%%%%%%%%%%%%%%%%%%%%%%%%%
In this section we employ the Tor specctral sequence to understand the functoriality of $\THH$ for CDVRs. See \cite[~Proposition 7.2.1.19]{lurie2017higher} for a statement of the Tor spectral sequence for module spectra over a ring spectrum.
%or \cite[\href{https://stacks.math.columbia.edu/tag/061Y}{Tag 061Y}]{stacks-project} for the more classical result in the case of chain complexes.
Let $R$ be a mixed characteristic complete discrete valuation ring with perfect residue field. The following are the main steps in the computation.
\begin{enumerate}
    \item Write $R$ as a pushout $R\cong\Z_p[z]\otimes_{\Z_p[z]}\Z_p$ via action through the Eisenstein polynomial
    \item $\THH$ preserves this pushout because the involved rings are commutative, and for $\Einf$-rings, $\THH$ is given by the colimit over $S^1$ in $\CAlg$ which means that it commutes with arbitrary colimits.
    \item Use the fact that $\THH(\Z_p[z])\simeq \THH(\Z_p)\otimes_\Z \HH(\Z[z]/\Z)$ but not naturally. But have natural isomorphism on homotopy groups
    $\THH_*(\Z_p[z])\simeq \THH_*(\Z_p)\otimes_\Z \Omega_{\Z[z]/\Z)}$.
    \item Now employ Tor-spectral sequence:\\
     $E^2_{i,j}=\Tor_{\pi_*(R)}^i\left( \pi_*(M), \pi_*(M)\right)_{(j)}\Rightarrow \pi_{i+j}\left( M\otimes_R N \right)$
    , i.e. in our case: 
    \\
    $\Tor_{\THH_*(\Z_p[z])}^i\left( \THH_*(\Z_p[z]),\THH_*(\Z_p)\right)_{(j)}\Rightarrow \THH_{i+j}(R)$ (or rather the $p$-completed version)
    \item Analyze the differentials and extension problems in the spectral sequence.
    \item Do there exist lifts?
\end{enumerate}
\begin{lem}
    For all (ordinary) rings $R$, the map $\THH(R;\Z_p)\to \HH(R;\Z_p)$ induces an natural isomorphism on the first $2p-1$ homotopy groups
    \begin{equation*}
        \THH_i(R;\Z_p)\rai \HH_i(R;\Z_p), \text{ for } i=0,1,\dots 2p-2
    \end{equation*}
\end{lem}
\begin{proof}
    Proposition \ref{tpovertp} allows us to write the map as
    \begin{equation*}
        \THH(R;\Z_p)\otimes_{\THH(\Z_p;\Z_p)}\THH(\Z_p;\Z_p)\to \THH(R;\Z_p)\otimes_{\THH(\Z_p;\Z_p)}\Z_p\stackrel{\ref{tpovertp}}{=}\HH(R;\Z_p).
    \end{equation*}
    We know by Proposition \ref{LM}, that the map $\THH(\Z_p;\Z_p)\to \Z_p$ is an isomorphism on $\pi_i$ for $i=0,\dots 2p-2$ and hence also $\THH(R;\Z_p)\to \HH(R;\Z_p)$ is, because tensoring can only increase connectivity.
\end{proof}
Together with the content of the last section, this functorially identifies the first $2p-1$ homotopy groups of $\THH(R;\Z_p)$. This is the best that we can hope for. In the next degree we only get a natural surjection but not an isomorphism, we e.g. have $\THH_{2p+1}(\Z_p;\Z_p)=\F_p$ but $\HH_{2p+1}(\Z_p;\Z_p)=0$. \\
For information on the remaining homotopy groups we proceed as follows.\\
We will crucially use Lemma \ref{description mixed char CDVR} to write all our rings as quotients of $\Z_p[z]$. This allows us to deduce our computations from the computation for $\THH(\Z_p[z])$ and to understand the functoriality of $\THH(\Z_p[z])$ under self maps of $\Z_p[z]$. We do not understand the functoriality of $\THH(\Z_p[z])$ on the level of spectra, but for homotopy groups we have a relative Hochschild-Kostant-Rosenberg style result.
\begin{lem}
    Let $R$ be an ordinary ring. Then we have the following expression for the topological Hochschild homology of the polynomial ring over $R$.
    \begin{equation*}
        \THH(R[z])\simeq \THH(R)\otimes_\Z \HH(\Z[z]).
    \end{equation*}
    This is not a natural equivalence. But the resulting isomorphism on homotopy groups $\THH_*(R[z])\simeq \THH_*(R)\otimes_\Z \Omega_{\Z[z]/\Z}$ is a natural isomorphism.
\end{lem}
\begin{proof} Using the basic properties of $\THH$ and the equivalence $R[z]\simeq R\otimes_\S \S[z]$ we obtain
    \begin{align*}
        \THH(R[z])\simeq \THH(R\otimes_\S \S[z])&\simeq\THH(R)\otimes_\S \THH(\S[z]) \\
        &\simeq \THH(R)\otimes_\Z (\Z\otimes \THH(\S[z])) \\
        &\simeq\THH(R)\otimes_\Z (\THH(\Z\otimes \S[z]/\Z))\\
        &\simeq 
         \THH(R)\otimes_\Z \HH(\Z[x])
         \qedhere
    \end{align*}
\end{proof}
In particular we have a natural isomorphism $\THH(\Z_p[z])\simeq \THH(\Z_p)\otimes_\Z \HH(\Z[z])$. Taking the $p$-completion yields 
\begin{equation*}
    \THH(\Z_p[z];\Z_p)\simeq \THH(\Z_p;\Z_p)\otimes_{\Z_p} \HH(\Z_p[z]/\Z_p).
\end{equation*} 
We do not need to $p$-complete on the outside of the tensor product. %because ...\todo{Write reason}.\\
Let us now apply this result. For this and for the rest of the section let $R$ be a mixed-characteristic complete discrete discrete valuation ring with perfect residue field of characteristic $p$. By Lemma \ref{description mixed char CDVR} we can thus find an Eisenstein polynomial $E(z)$ such that
\begin{equation*}
    R=\Z_p[z]/E(z)=\Z_p[z]\otimes_{\Z_p[z]}\Z_p,
\end{equation*}
with actions prescribed by the maps $\Z_p[z]\ra \Z_p[z], z\mapsto E(z)$ and $\Z_p[z]\to \Z_p, z\mapsto 0$. \\
Using the monoidality of $\THH$ (Proposition 2.6) we therefore get 
\begin{equation*}
    \THH(R)=\THH(\Z_p[z])\otimes_{\THH(\Z_p[z])}\THH(\Z_p).
\end{equation*}
To $p$-complete the left hand side, we again only have to $p$-complete all the involved terms and not on the outside 
%because ...\todo{Write reason}. 
We then know all the homotopy groups of the terms involved on the right side, because we know $\THH(\Z_p[z];\Z_p) $ and $\THH(\Z_p;\Z_p)$. But we of course want to know the homotopy groups of $\THH(R;\Z_p)$ and for this we can use the Tor-spectral sequence. 
It takes the form
\begin{equation*}
    E^2_{i,j}=\Tor_{\THH_*(\Z_p[z];\Z_p)}^i\left( \THH_*(\Z_p[z];\Z_p),\THH_*(\Z_p;\Z_p)\right)_{(j)}\Rightarrow \THH_{i+j}(R;\Z_p)
\end{equation*}
\textbf{The $E^2$-page only contains terms we functorially know}.

%This will allow us to compute the effect of $\THH$ on maps $R\to S$.
Since the action of $\THH_*(\Z_p[z];\Z_p)$ on $\THH_*(\Z_p;\Z_p)$ factors through the action of $\HH(\Z_p[z];\Z_p)=\Omega^*_{\Z_p[z]/\Z_p}$ on $\Z_p$, it suffices to resolve $\Z_p$ as a $\Omega^*_{\Z_p[z]/\Z_p}$-module. A resolution can be provided by a free divided power algebra on an exterior algebra over $\Omega^*_{\Z_p[z]/\Z_p}$, i.e. we have a quasi-isomorphism of graded $\Omega^*_{\Z_p[z]/\Z_p}$-modules
\begin{equation*}
    \Z_p \lai \Gamma_{\Lambda_{\Omega^*_{\Z_p[z]/\Z_p}}(b)}\{a\}, \ |a|=(1,0), \ |b|=(1,1), \ \partial(a)=dz, \ \partial(b)=z.
\end{equation*}
Explicitely the resolution looks as follows:
\begin{align*}
    \Z_p\la \Omega^*_{\Z_p[z]/\Z_p}\la \Omega^*_{\Z_p[z]/\Z_p}\{a,b\}\la \Omega^*_{\Z_p[z]/\Z_p}\{a^{[2]}, ba\}\la \dots 
\end{align*}
with the maps (from left to right) given by  $[0\mapsfrom z, dz]$, $[z\mapsfrom b, dz\mapsfrom a]$ and $[az-bdz \mapsfrom ba, adz \mapsfrom a^{[2]}]$.
Let us abbreviate $\Gamma\coloneqq \Gamma_{\Lambda_{\Omega^*_{\Z_p[z]/\Z_p}}(b)}\{a\}$. The $E^2$-page is then the homology of
\begin{equation*}
    \THH(\Z_p;\Z_p)\otimes 
        \left(   
        \Omega^*_{\Z_p[z]/\Z_p} \otimes_{\Omega^*_{\Z_p[z]/\Z_p}} \Gamma   
        \right).
\end{equation*} 
The tensor product $\Omega^*_{\Z_p[z]/\Z_p} \otimes_{\Omega^*_{\Z_p[z]/\Z_p}}-$ has only the effect, that it changes the differential to $\partial(a)=E'(z)dz, \ \partial(b)=E(z)$. Let us denote this dga by $\Gamma'$. Because $\THH_i(\Z_p;\Z_p)$ is only nonzero in degrees 0 and $i=2p^kl-1$, where it is given by $\Z/p^k$.
Hence to fully compute the $E^2$-page, we must first identify the homology of $\Gamma'$ and then take the direct sum of this homology with infinitely many copies of shifts of its derived mod $p^n$ reduction, i.e. for each $k,l\geq 1$ and $p\nmid l$ we add another copy of the mod $p^k$ reduction of the homology of $\Gamma'$ shifted by $(2p^kl-1)$.\\
Finish description of $E^2$-page, i.e. give computation of homology of $\Gamma'$.
\\
We can then calculate the effects of maps $\varphi:R\to S$ on $\THH$ via this spectral sequence \textit{provided} that we can find lifts $f,g$ in the following diagram:
% https://q.uiver.app/?q=WzAsNixbMCwyLCJSIl0sWzEsMiwiUyJdLFswLDEsIlxcWl9wW3pdIl0sWzEsMSwiXFxaX3Bbd10iXSxbMCwwLCJcXFpfcFtyXSJdLFsxLDAsIlxcWl9wW3NdIl0sWzAsMSwiXFx2YXJwaGkiXSxbMiwwLCJ6XFxtYXBzdG8gXFxwaV9SIiwyXSxbMywxLCJ3XFxtYXBzdG8gXFxwaV9TIl0sWzQsMiwiclxcbWFwc3RvIEVfUih6KSIsMl0sWzUsMywic1xcbWFwc3RvIEVfUyh3KSJdLFsyLDMsImYiLDAseyJzdHlsZSI6eyJib2R5Ijp7Im5hbWUiOiJkYXNoZWQifX19XSxbNCw1LCJnIiwwLHsic3R5bGUiOnsiYm9keSI6eyJuYW1lIjoiZGFzaGVkIn19fV1d
\[\begin{tikzcd}
	{\Z_p[r]} & {\Z_p[s]} \\
	{\Z_p[z]} & {\Z_p[w]} \\
	R & S
	\arrow["\varphi", from=3-1, to=3-2]
	\arrow["{z\mapsto \pi_R}"', from=2-1, to=3-1]
	\arrow["{w\mapsto \pi_S}", from=2-2, to=3-2]
	\arrow["{r\mapsto E_R(z)}"', from=1-1, to=2-1]
	\arrow["{s\mapsto E_S(w)}", from=1-2, to=2-2]
	\arrow["f", dashed, from=2-1, to=2-2]
	\arrow["g", dashed, from=1-1, to=1-2]
\end{tikzcd}\] 
The first lift can always be found, but the second one is unfortunately not always possible\footnote{For example the interesting case $\Z_p[\zeta_p]\to \Z_p[\zeta_p],\ \zeta_p\to \zeta_p^2$ is not covered.}, but sometimes it is. The applicability is mostly disjoint \todo{Actually not, apparently the Tor spectral sequence covers morst cases, in which the descent SS can be used } from the method that we will outline in the next section. 
Some examples in which we can apply the method are:
\begin{itemize}
    \item The inclusion $\Z_p\hookrightarrow \Z_p[\sqrt{p}]$ 
    \item Or more generally $\Z_p[\sqrt[k]{p}]\hookrightarrow \Z_p[\sqrt[kn]{p}]$
    \item $\Z_p[\sqrt{p}]\to \Z_p[\sqrt{p}], \sqrt{p}\mapsto -\sqrt{p}$
    \item Or more generally $\Z_p[\sqrt[2n]{p}]\to \Z_p[\sqrt[2n]{p}], \sqrt[2n]{p}\mapsto -\sqrt[2n]{p}$
    \item $\Z_p[\sqrt[p-1]{p}]\to \Z_p[\sqrt[p-1]{p}], \sqrt[p-1]{p}\mapsto \zeta_{p-1}\sqrt[p-1]{p}$, for $p>2$
    (Recall that $\Z_p$ has all $(p-1)$-th roots of unity by Hensel's Lemma)
\end{itemize}
 \todo{Explicit computations and extension/differential problems}
%%%%%%%%%%%%%%%%%%%%%%%%%%%%%%%%%%%%%%%%%%%%%%%%%%%%%%%%%%%%%%%%%%%
\section{(Partial) Functoriality of $\THH$ via the descent spectral sequence}
%%%%%%%%%%%%%%%%%%%%%%%%%%%%%%%%%%%%%%%%%%%%%%%%%%%%%%%%%%%%%%%%%%%
Let us try to mimick, the procedure of Section \ref{HHFunctoriality} for \textit{topological} Hochschild homology. We can define the category of \textbf{CDVRs with spherical lifts }
$\CDVRSL$ as follows. Objects are again pairs $(R,\pi)$, where $R$ is a complete discrete valuation ring of mixed characteristic with residue field $\F_p$ and $\pi\in R$ is a uniformizer. The element $\pi$ determines a map $\S[z]\to R,\ z\to \pi$. A morphism from $(R,\pi)$ to $(S,\epsilon)$ consists of a pair $(\phi,f)$, where $\phi:R\to S$ is a $\Z_p$-algebra homomorphism, $f:\S[z]\to \S[w]$ a map of $\Einf$-ring spectra, such that the following square commutes:
% https://q.uiver.app/?q=WzAsNCxbMCwwLCJcXFNbel0iXSxbMSwwLCJcXFNbd10iXSxbMCwxLCJSIl0sWzEsMSwiUyJdLFswLDEsImYiXSxbMiwzLCJcXHBoaSJdLFswLDJdLFsxLDNdXQ==
\[\begin{tikzcd}
	{\S[z]} & {\S[w]} \\
	R & S
	\arrow["f", from=1-1, to=1-2]
	\arrow["\phi", from=2-1, to=2-2]
	\arrow[from=1-1, to=2-1]
	\arrow[from=1-2, to=2-2]
\end{tikzcd}\]
This category is the natural home for relative $\THH$ computations. We have forgetful functors $\CDVRSL\to \CDVRL\to \CDVRu$. The problem is, that the first functor is \textit{not} full. We cannot lift an arbitrary polynomial map $\Z_p[z]\to \Z_p[w]$ to a map of ring spectra $\S[z]\to \S[w]$. This is only\todo{check this} possible for \textbf{monomial maps}, i.e. maps of the form $\Z_p[z]\to \Z_p[w],\ z\to w^n$. In this specific case, we can define the lift $\S[z]\to \S[w]$ via applying $\sip$ to the map of monoids $\N\xrightarrow{\cdot n} \N$.
\\
%$\mathrm{CDVR}^{\S-\mathrm{Lift}}$.
The descent spectral sequence of section 3.4 does have a partial functoriality. Let $R$ and $S$ be two CDVRs with chosen uniformizers $\pi_R$ and $\pi_S$, which provide them with respective module structures over $\S[z]$. We then have functoriality in maps $\varphi:R\to S$ that lifts to a monomial map of the spherical polynomial ring\footnote{induced by the map $\N\xrightarrow{\cdot n} \N$}, i.e. $\S[z]\to \S[w],\  z\mapsto w^n$, such that the following square commutes
% https://q.uiver.app/?q=WzAsNCxbMCwwLCJSIl0sWzEsMCwiUyJdLFswLDEsIlxcU1t6XSJdLFsxLDEsIlxcU1t6XSJdLFswLDEsIlxcdmFycGhpIl0sWzIsMCwielxcbWFwc3RvIFxccGlfUiJdLFsyLDMsInpcXG1hcHN0byB6Xm4iLDJdLFszLDEsInpcXG1hcHN0byBcXHBpX1MiLDJdXQ==
\[\begin{tikzcd}
	R & S \\
	{\S[z]} & {\S[z]}
	\arrow["\varphi", from=1-1, to=1-2]
	\arrow["{z\mapsto \pi_R}", from=2-1, to=1-1]
	\arrow["{z\mapsto w^n}"', from=2-1, to=2-2]
	\arrow["{w\mapsto \pi_S}"', from=2-2, to=1-2]
\end{tikzcd}\]

This applies for example in the cases of the inclusions $\Z_p\hookrightarrow \Z_p[\sqrt[n]{p}]$ or $\Z_p[\sqrt[k]{p}]\hookrightarrow \Z_p[\sqrt[kn]{p}]$. \\
Using the base change formula 
\begin{equation*}
    \THH(R/\S[z])=\THH(R)\otimes_{\THH(\S{z]})}\S[z]
\end{equation*}
We obtain the following map of pushout squares and thus get the dashed induced map.
% https://q.uiver.app/?q=WzAsOCxbMCwzLCJcXFRISChcXFNbel0pIl0sWzIsMywiXFxUSEgoUi9cXFNbel0pIl0sWzAsMSwiXFxUSEgoUikiXSxbMiwxLCJcXFNbel0iXSxbMSwwLCJcXFRISChTKSJdLFszLDAsIlxcU1t3XSJdLFszLDIsIlxcVEhIKFMvXFxTW3ddIl0sWzEsMiwiXFxUSEgoXFxTW3ddKSJdLFswLDFdLFsyLDBdLFsyLDNdLFszLDFdLFsyLDRdLFszLDVdLFsxLDYsIiIsMCx7InN0eWxlIjp7ImJvZHkiOnsibmFtZSI6ImRhc2hlZCJ9fX1dLFs1LDZdLFswLDddLFs3LDZdLFs0LDddLFs0LDVdXQ==
\[\begin{tikzcd}
	& {\THH(S)} && {\S[w]} \\
	{\THH(R)} && {\S[z]} \\
	& {\THH(\S[w])} && {\THH(S/\S[w])} \\
	{\THH(\S[z])} && {\THH(R/\S[z])}
	\arrow[from=4-1, to=4-3]
	\arrow[from=2-1, to=4-1]
	\arrow[from=2-1, to=2-3]
	\arrow[from=2-3, to=4-3]
	\arrow[from=2-1, to=1-2]
	\arrow[from=2-3, to=1-4]
	\arrow[dashed, from=4-3, to=3-4]
	\arrow[from=1-4, to=3-4]
	\arrow[from=4-1, to=3-2]
	\arrow[from=3-2, to=3-4]
	\arrow[from=1-2, to=3-2]
	\arrow[from=1-2, to=1-4]
\end{tikzcd}\]
We get a map of the spectral sequences for $R$ and $S$, which is on the $E^2$-pages given by:
\begin{equation*}
    R[z]\otimes_{\Z[z]} \Omega^*_{\Z[z]}\xrightarrow{\varphi\otimes df} R[z]\otimes_{\Z[w]} \Omega^*_{\Z[w]},
\end{equation*} 
where $f:\Z[z]\to\Z[w], z\mapsto w^k$. \\
From the spectral sequence we get the following short exact sequence for the odd homotopy groups of $\THH(R;\Z_p)$: 
\begin{equation*}
    0\to R\{x^{n+1}\}\xrightarrow{d^2} R\{x^{n}dz\}\twoheadrightarrow \THH_{2n+1}(R;\Z_p)\to 0
\end{equation*}
Now given a map $\varphi:R\to S$, that makes the square commute with a monomial map we obtain an induced map of short exact sequences:

% https://q.uiver.app/?q=WzAsMTAsWzAsMCwiMCJdLFsxLDAsIlJcXHt4XntuKzF9XFx9Il0sWzIsMCwiUlxce3hee259ZHpcXH0iXSxbMywwLCJcXFRISF97Mm4rMX0oUjtcXFpfcCkiXSxbNCwwLCIwIl0sWzAsMSwiMCJdLFsxLDEsIlNcXHt4XntuKzF9XFx9Il0sWzIsMSwiU1xce3hee259ZHpcXH0iXSxbMywxLCJcXFRISF97Mm4rMX0oMjtcXFpfcCkiXSxbNCwxLCIwIl0sWzAsMV0sWzEsMiwiZF4yIl0sWzIsMywiIiwwLHsic3R5bGUiOnsiaGVhZCI6eyJuYW1lIjoiZXBpIn19fV0sWzMsNF0sWzUsNl0sWzYsNywiZF4yIl0sWzcsOCwiIiwwLHsic3R5bGUiOnsiaGVhZCI6eyJuYW1lIjoiZXBpIn19fV0sWzgsOV0sWzEsNiwiXFx2YXJwaGkiXSxbMiw3LCJcXHZhcnBoaVxcb3RpbWVzIGRmIl0sWzMsOCwiIiwxLHsic3R5bGUiOnsiYm9keSI6eyJuYW1lIjoiZGFzaGVkIn19fV1d
\[\begin{tikzcd}
	0 & {R\{x^{n+1}\}} & {R\{x^{n}dz\}} & {\THH_{2n+1}(R;\Z_p)} & 0 \\
	0 & {S\{x^{n+1}\}} & {S\{x^{n}dw\}} & {\THH_{2n+1}(S;\Z_p)} & 0
	\arrow[from=1-1, to=1-2]
	\arrow["{d^2}", from=1-2, to=1-3]
	\arrow[two heads, from=1-3, to=1-4]
	\arrow[from=1-4, to=1-5]
	\arrow[from=2-1, to=2-2]
	\arrow["{d^2}", from=2-2, to=2-3]
	\arrow[two heads, from=2-3, to=2-4]
	\arrow[from=2-4, to=2-5]
	\arrow["\varphi", from=1-2, to=2-2]
	\arrow["{\varphi\otimes df}", from=1-3, to=2-3]
	\arrow[dashed, from=1-4, to=2-4]
\end{tikzcd}\]
We see, that the functoriality is fully encoded in the explicit map $\varphi\otimes df: R\{x^{n}dz\}\to S\{x^{n}dw\}, \ rx^ndz\mapsto \varphi(r)f'(\pi_S)x^ndw$.\todo{compare to previous sections}
%In the simplest example of the inclusion $\Z_p\hookrightarrow \Z_p[\sqrt{p}]$, the map 