\appendix
\chapter{K-Theory}
In this appendix we give a brief review of the definition of algebraic K-theory for (connective) ring spectra, e.g. ordinary rings. The construction can be performed as the composition of four intermediate steps, which we will outline now. \\
For this, let $R$ be a connective ring spectrum. \\
1) An $R$-module spectrum $M$ is called \textbf{finitely generated projective}, if is a retract of a free $R$-module, i.e. there is another $R$-module $N$ and  $n\in\N$ such that $M\oplus N=R^n$.
Denote by $\Proj_R\subset \Mod_R$ the (full) subcategory of finitely generated projective $R$-module spectra. This is a symmetric monoidal $\infty$-category under the direct sum. \\
2) For any $\infty$-category $\cC$ we can take its \textbf{groupoid core} $\cC^{\simeq}$, which is the largest subcategory of $\cC$ containing only equivalences. This is an $\infty$-groupoid or equivalently a space. If $\cC$ has a symmetric monoidal structure, it induces an $\Einf$-structure on $\cC^{\simeq}$. In particular $(\Proj_R)^{\simeq}$ is an $\Einf$-space.\\
3) Next we need the concept of an \textbf{group-like $\Einf$-space}, which is simply an $\Einf$-space $X$, such that $\pi_0(X)$ is a group under the monoid operation induced by the $\Einf$-structure. We define the category of group-like $\Einf$-spaces $\CAlg_{\gp}(\Spc)$ as the full subcategory of $\CAlg(\Spc)$ containing those  $\Einf$-spaces, that are group like. The crucial fact is that the thus provided inclusion $\CAlg_{\gp}(\Spc)\ra \CAlg(\Spc)$ admits a left adjoint $X\to X^{\gp}$, called \textbf{group completion}. On the $\pi_0$ level, it actually induces the ordinary group completion which provides a initial abelian group for any commutative monoid. The existence of the adjoint can either be proved using the adjoint functor theorem or providing a concrete construction, like $X^{\gp}=\Omega BX$. \\
4) The last step uses the equivalence of $\infty$-categories between connective spectra and grouplike $\Einf$-spaces: $\Sp_{\geq 0}\simeq \CAlg_{\gp}(\Spc)$. One functor is provided by $X\ra \Omega^{\infty}X$. Let us denote the other one by $B^\infty:\CAlg_{\gp}(\Spc)\rai \Sp_{\geq 0} $.  \\
These four definitions/observations now allow us to define the K-theory spectrum $K(R)$ of a connective ring spectrum $R$ as the spectrum corresponding to the group completion of the groupoid core of the symmetric monoidal $\infty$-category of projective $R$-modules. 
\begin{align*}
        K: R\mapsto \Proj_R\mapsto \Proj_R^{\simeq}\mapsto (\Proj_R^{\simeq})^{\gp}\mapsto B^{\infty}(\Proj_R^{\simeq})^{\gp}\in \Sp
\end{align*}
\begin{rmk}
    If $R$ is a commutative ring spectrum, $K(R)$ also is, more precisely for an $\En$-ring spectrum $R$, $K(R)$ is a connective $\mathbb{E}_{n-1}$-ring spectrum. In the language, used here, this is proved in \cite{GGNuniversalityloop}. The last chapter in there also contains more details on the above construction of the K-theory spectrum and proofs of the claimed facts.
\end{rmk}
\begin{rmk}
    In the case of a non-connective ring spectrum $R$, this definition is no longer suitable and we need to proceed differently. In this case we should not consider $\Proj_R$ but $\Perf_R$ instead. This is the smallest stable subcategory of $\Mod_R$ containing $R$ and which is closed under retracts. The $\infty$-category $\Perf_R$ has the structure of a Waldhausen $\infty$-category in the sense of Barwick and we can define K-theory for it, see \cite[Chapter~11]{barwickK-theory}. This construction is also suitably monoidal by \cite[Proposition~3.8]{barwick2013multiplicative}. In particular we again have the statement that $K(R)$ is an $\mathbb{E}_{n-1}$-ring spectrum for an $\En$-ring spectrum $R$, 
\end{rmk}
