\chapter{Functoriality for THH of CDVRs}
In this chapter, we will use the Tor-spectral sequence to understand the effect of THH on maps between complete discrete valuation rings.
See \cite[~Proposition 7.2.1.19]{lurie2017higher} for a statement of the Tor spectral sequence for module spectra over a ring spectrum 
or \cite[\href{https://stacks.math.columbia.edu/tag/061Y}{Tag 061Y}]{stacks-project} for the more classical result in the case of chain complexes.
\\
Let $R$ be a mixed characteristic complete discrete valuation ring with perfect residue field. The following are the main steps in our proof.
\begin{enumerate}
    \item Write $R$ as a pushout $R\cong\Z_p[z]\otimes_{\Z_p[z]}\Z_p$ via action through the Eisenstein polynomial (see footnote 5)
    \item $\THH$ preserves this pushout because the involved rings are commutative, and for $\Einf$-rings, $\THH$ is given by the colimit over $S^1$ in $\CAlg$ which means that it commutes with arbitrary colimits.
    \item Use the fact that $\THH(\Z_p[z])\simeq \THH(\Z_p)\otimes_\Z \HH(\Z[z]/\Z)$ but not naturally. But have natural isomorphism on homotopy groups
    $\THH*(\Z_p[z])\simeq \THH_*(\Z_p)\otimes_\Z \Omega_{\Z[z]/\Z)}$.
    \item Now employ Tor-spectral sequence:\\
     $E^2_{i,j}=\Tor_{\pi_*(R)}^i\left( \pi_*(M), \pi_*(M)\right)_{(j)}\Rightarrow \pi_{i+j}\left( M\otimes_R N \right)$
    , i.e. in our case: 
    \\
    $\Tor_{\THH_*(\Z_p[z])}^i\left( \THH_*(\Z_p[z]),\THH_*(\Z_p)\right)_{(j)}\Rightarrow \THH_{i+j}(R)$ (or rather the $p$-completed version)
    \item Analyze the differentials and extension problems in the spectral sequence.
\end{enumerate}
\section{Functoriality of $\HH$ for CDVRs}
\begin{enumerate}
    \item Write down pushout square, apply HH to it
    \item Need to compute derived TP, 
    \item Resolve $\Z_p$ over exterior algebra $\Lambda_{\Z_p}(e)$ gives divided power algebra
    \item TP then gives same DP algebra but with different differential
    \item Compute Homology of this, concentrated in odd degrees and given by the same group
    \item But different functoriality, power.
\end{enumerate}
First describe functoriality of Hochschild homology.
